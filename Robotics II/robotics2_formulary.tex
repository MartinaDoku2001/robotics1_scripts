\documentclass[a4paper,12pt]{article}

% Pacchetti di base
\usepackage[utf8]{inputenc}
\usepackage{amsmath, amssymb}
\usepackage{graphicx}
\usepackage{hyperref}
\usepackage{geometry}
\usepackage{fancyhdr}
\usepackage{titlesec}
\usepackage{tocloft}
\usepackage{enumitem}

% Margini del documento
\geometry{left=2.5cm, right=2.5cm, top=3cm, bottom=3cm}

% Impostazioni del header e footer
\pagestyle{fancy}
\fancyhf{}
\fancyhead[L]{\leftmark}
\fancyhead[R]{\thepage}

% Definizione della copertina
\title{
    \Huge{Robotics 2}\\
    \large{Artificial Intelligence and Robotics}\\
    \vspace{1cm}
    \includegraphics[width=0.4\textwidth]{logo.png}
}
\author{Martina Doku \and Giuseppina Iannotti \and Gianmarco Scarano}

% Formattazione dei titoli delle sezioni
\titleformat{\section}{\Large\bfseries}{\thesection}{1em}{}
\titleformat{\subsection}{\large\bfseries}{\thesubsection}{1em}{}
\titleformat{\subsubsection}{\normalsize\bfseries}{\thesubsubsection}{1em}{}

% Formattazione dell'indice
\renewcommand{\cftsecfont}{\bfseries}
\renewcommand{\cftsubsecfont}{\itshape}
\renewcommand{\cftsubsubsecfont}{\small}

\begin{document}

% Copertina
\maketitle
\newpage

% Indice
\tableofcontents
\newpage

\section{Calibration}
\textbf{Calibration goal}:  Recover as 
much as possible pose
errors by correcting the nominal \ref{glossary:nominal condition}
set of D-H parameters,
based on independent external measurements
\subsection{Error measure}
The error measure is the difference between the true value $r$ and the
 measured value $r_{nom}$. It is given by:
\begin{equation}
  \Delta r = r - r_{nom} = \frac{\partial f}{\partial \alpha} \Delta \alpha + \frac{\partial f}{\partial a} \Delta a + \frac{\partial f}{\partial d} \Delta d + \frac{\partial f}{\partial \theta} \Delta \theta
\end{equation}
where $\frac{\partial f}{\partial \alpha}$, $\frac{\partial f}{\partial a}$, $\frac{\partial f}{\partial d}$, $\frac{\partial f}{\partial \theta}$ are the partial 
jacobians evaluated in nominal conditions
\subsection{Calibration Equation}
The calibration equation is given by:
\begin{equation}
    \Delta \psi = \begin{bmatrix}
        \Delta \alpha \\
        \Delta a \\
        \Delta d \\
        \Delta \theta
    \end{bmatrix}
\end{equation}
\begin{equation}
    \phi= \begin{bmatrix}
     \frac{\partial f}{\partial \alpha} & \frac{\partial f}{\partial a} & \frac{\partial f}{\partial d} & \frac{\partial f}{\partial \theta}
    \end{bmatrix}
\end{equation}
\begin{equation}
    \Delta r= \phi \cdot \Delta \psi
\end{equation}
\textbf{Note:} If we have more unknowns than equations, we need to perform 
$l$ experiments to solve the calibration equation. Thus we obtain:
\begin{equation}
    \Delta \bar{r}= \bar{\phi} \cdot \Delta \psi
\end{equation}
where $\Delta \bar{r}$ is the average error 
and $\bar{\phi}$ is the regressor matrix evaluated at nominal values,
and $\Delta \psi$ is the unknowns.
\subsection{Calibration Algorithm}
The calibration algorithm is performed using the following steps:
\begin{enumerate}
    \item $\Delta \bar{r}= \bar{\phi} \cdot \Delta \psi$
    \item $\Delta \psi = \bar{\phi}^\# \Delta\bar{r}= (\bar{\phi}^T \cdot \bar{\phi})^{-1} \cdot \bar{\phi}^T \cdot \Delta \bar{r}$
    \item $ \psi_{nom} + \Delta \psi = \psi ' $
    \item $ \Delta \bar{r}' = \bar{\phi}' \cdot \Delta \psi$ where $\bar{\phi}'$ is the regressor
     matrix evaluated at the new values.
    \item repeat from step 2
\end{enumerate}
\textbf{Note:} we use the pseudo-inverse to solve the calibration equation,
it returns the minimum norm solution in case of underdetermined system,
the minimum error solution in case of overdetermined system.
\subsubsection{ex: Write regressor matrix from a DH table}
The first step is to obtain the position vector from the DH table.
Then, for each component (a, $\alpha$, d, $\theta$) that we are interested in, we
build the jacobian $ \frac{\partial f}{\partial \alpha} $... .
The regressor equation is then given by:
\begin{equation}
    \Delta p = J_a \Delta a + J_{\alpha} \Delta \alpha + J_d \Delta d + J_{\theta} \Delta \theta= \Phi \Delta \phi
\end{equation}
with:
\begin{equation}
    \Phi = \begin{bmatrix}
        J_a & J_{\alpha} & J_d & J_{\theta}
    \end{bmatrix}
\end{equation}
\textbf{Note:} if we are interested ony in the fine tuning of some parameters, we can
exclude the others from all the equations.
\subsubsection{ex: Find optimal link length from positions and joints}
The steps are the following:
\begin{enumerate}
    \item Compute the position vector from the forward kinematics on the given 
    joint measures and expected link lengths (if given)
    \item Check if the position vector is the same as the one
     measured, if not go to step 3
    \item rewrite the position vector as a function of the link 
    lengths:\\ 
    $\Delta p = p - p_{nom} = \phi(q) \Delta l$ \\
    $\Delta p = \begin{bmatrix}
        \Delta p_a\\
        \Delta p_b \\
        \Delta p_c \\ \Delta p_d
    \end{bmatrix} =
    \begin{bmatrix}
        \phi (q_a)\\
         \phi(q_b) \\
          \phi(q_c) \\ \phi(q_d)
    \end{bmatrix} \Delta l$ 
    \textbf{Note:} $\Delta p_a$ and co. are vertical vectors, they must 
    be stacked vertically (if i have 3 measures in 2 dim, i obtain a 6x1 vector)
    \item where a, b, c, d are the different joint measures 
    \item solve the calibration equation for l as: $\Delta l =  \Phi^{\#} \Delta p$
\end{enumerate}
\subsubsection{ex: Theoretical question}
\textbf{Question}: When is it convenient to use a two steps calibration?\\
\textbf{Answer}: It is convinient to use a two steps calibration when it is expected
 that the DH parameters will have very different uncertainty ranges. \\
 The first step is performed only on high uncertainty parameters, the second on all of them.
 This improves the accuracy of the pseudoinverse of the regressor matrix





\section{Inverse Kinematics with Redundancy}
Types of redundancy:
\begin{itemize}
    \item \textbf{kinematic redundancy}: the robot has more degrees of freedom than needed
    \item \textbf{components redundancy}: the robot has more
     components than needed
     \item \textbf{control redundancy}: the robot is redundant in its supervision
     architecture
\end{itemize}
    The resolution of redundancy can be done using two methods:
\begin{enumerate}
    \item \textbf{Local methods}: the local methods are online methods
     that are used to solve the redundancy problem. They are Linear Quadratic problems.
     (discrete approach)
    \item \textbf{Global methods}: the global methods are offline methods.
     They are used to solve the redundancy problem. They are a
      non-linear problem.
\end{enumerate}
\subsection{Which method to choose?}
If the exercise asks for the solution that:
\begin{itemize}
    \item \textbf{minimizes the kinetic energy}: use the weighted pseudo-inverse with the inertia matrix as weight
    \item \textbf{minimizes the torque norm (or task error norn)}: use the pseudo-inverse
    \item \textbf{minimize norm of joint velocities}: use the pseudo-inverse
    \item \textbf{minimizes the error on the higher priority task}: use the task priority (recursive) method
    \item \textbf{minimizes the error on the lower priority task while correctly executing the higher priority one}: use the nullspace method or task priority (recursive) method (
        the second is better if you have more than 2 tasks
    )
    \item \textbf{minimizes the acceleration norm}: use the nullspace method as in \ref{sec:Resolution at acceleration level}
\end{itemize}
\textbf{Note:} if asked to find another solution without 
specific requirements, it is usually a solution that you can 
infer by looking
at the jacobian and the desired velocity, without 
the need of any specific method.
\subsection{Local methods}
The three kinds of local methods are:
\begin{enumerate}
    \item \textbf{Jacobian-based methods}: among the infinite solutions,
     one is chosen, e.g., that minimizes a
    suitable (possibly weighted) norm. The problem of this method is that it 
    might encounter singularities and it can lead to non repeatable position 
    in joint space.
    \item \textbf{Null-space methds}: a term is added to the previous 
    solution so as not to affect execution of the task trajectory
    \item \textbf{Task-augmentation methods}: redundancy is 
    reduced/eliminated by adding $S \leq N - M$ additional tasks
\end{enumerate}
\subsection{Jacobian-based methods}
The Jacobian-based methods are used to solve the redundancy problem.
They give a solution in the form: \begin{equation}
    \dot{q} = K(q) \cdot \dot{r}
\end{equation}
where K is a generalized inverse of the Jacobian matrix. Possible 
choices for K are:
\begin{itemize}
    \item Pseudo-inverse \ref{sec:Pseudo-inverse}
    \item Weighted pseudo-inverse \ref{sec:Pseudo-inverse}
    \item Damped Least Square pseudo-inverse \ref{sec:Damped Least Square pseudo-inverse}:
     this method is used to avoid singularities.
\end{itemize}
    \subsubsection{Pseudo-inverse}\label{sec:Pseudo-inverse}
If J is full rank, the pseudo-inverse is given by:
\begin{equation}
    J^\# = J^T(J^T J)^{-1} 
\end{equation}
If J is not full rank, the pseudo-inverse it 
is computed by using the SVD decomposition (or \verb|pinv| in Matlab).\\
\textbf{Note:} the pseudo-inverse is not unit independent.
\subsubsection{Weighted pseudo-inverse}\label{sec:Weighted pseudo-inverse}
The weighted pseudo-inverse is given by:
\begin{equation}
    J^\# = W^{-1} J^T(J W^{-1} J^T)^{-1}
\end{equation}
where larger weights are given to the most important tasks, usally 
the weights are proportional to the inverse of the joint ranges.\\
\textbf{Note:} It is NOT a pseudoinverse but it shares some properties \ref{sec:Pseudo-inverse properties}.\\
\textbf{Note:} To find the velocity that \textbf{minimizes the Kinetic energy}, we have to
use the weighted pseudo-inverse with the \textbf{Inertia matrix} as weight.

\subsubsection{Damped Least Square pseudo-inverse}\label{sec:Damped Least Square pseudo-inverse}
The damped least square pseudo-inverse is given by:
\begin{equation}
    J^\# = J^T(J J^T + \mu^2 I)^{-1}
\end{equation}
where $\mu$ is a damping factor (the larger $\mu$, the larger the error)\\
It is a compromise between
large joint velocity
and task accuracy, it's used to obtain \textbf{robust behavior in the
 presence of singularities},
but in its basic version it always gives a task error.
\subsection{Null-space methods}
The null space methods are used to solve the redundancy problem.
They give a solution in the form: \begin{equation}
    \dot{q} = J^\# \cdot \dot{r} + (I - J^\# J) \cdot \dot{q}_0
\end{equation}
where $\dot{q}_0$ is a generic preferred joint velocity.\\
The equation is the sum of the solution of the primary task 
and the orthogonal projection of $\dot{q}_0$ on the null space of $J$ (The null space 
$\mathcal{N}$ represents the space of joint velocities that do not affect the end-effector position)
We have mainly two methods:
\begin{itemize}
    \item \textbf{Projected gradient method} $\Rightarrow$ \ref{sec:Projected gradient method}
    \item \textbf{Reduced gradient method} $\Rightarrow$ \ref{sec:Reduced gradient method}
\end{itemize}
    \subsubsection{Projected gradient method}\label{sec:Projected gradient method}
    The solution for the projected gradient method is obtained by the minimization 
    of the following function:
    \begin{equation}
        H = \frac{1}{2} (\dot{q}-\dot{q}_0)^T W (\dot{q}-\dot{q}_0)
    \end{equation}
    The solution is
     given by:\begin{equation}
        \dot{q} = J^\# \cdot \dot{r} + (I - J^\# J) \cdot \dot{q}_0
    \end{equation} or \begin{equation}
        \dot{q} = J^\# \cdot \dot{r} + (I - J^\# J) \cdot \nabla H(q)
    \end{equation}
    \textbf{Note:} there is a necessary condition of constrained 
    optimality: $(I - J^\# J) \nabla H(q) = 0$\\
    \textbf{Possible values of H} \\
    The possible values of H are:
    \begin{itemize}
        \item manipulanility: $H = \sqrt{det(J J^T)}$
        \item joint limits: $H_{range} = \frac{1}{2N}\sum_{i=1}^{n} \frac{(q_i - \bar{q_i})^2}{(q_{max_i} - q_{min_i})^2}$
        \textbf{Note:} In this case $\dot{q}_0 = - \nabla H_{range}$, we introduce the minus because
        we want to \texttt{minimize} the distance from the center of the joint range.
        \item obstacle avoidance: $H = \min\limits_{a \in robot} \min\limits_{b \in obstacle} \|a(q) - b\|^2$
    \end{itemize}

    \subsubsection{Reduced gradient method}\label{sec:Reduced gradient method}
     The solution is
     given by: \begin{equation}
        \dot{q} = \begin{bmatrix}
            \dot{q}_a \\
            \dot{q}_b
        \end{bmatrix} = \begin{bmatrix}
            J_a^{-1} \\
            0 
        \end{bmatrix}\dot{r}
        + \begin{bmatrix}
            -J_a^{-1} J_b \\ 
            I
        \end{bmatrix} \cdot \begin{bmatrix} 
            -(J_a^{-1}J_b)^T & I
        \end{bmatrix} \nabla _q H
    \end{equation}
    Where $J_a$ is the greatest non singular minor of J and $J_b$ is 
    the other minor. \\
    \textbf{Finding the non singular minor} \\
    The non singular minor can be found by using the following steps:
    \begin{enumerate}
        \item Compute the Jacobian J 
        \item Compute the rank r of J
        \item Compute the determinant of all submatrices of J of size r
        \item Select the minor with the greatest (absolute) determinant (it must be 
        non singular, so the determinant must be different from 0)
    \end{enumerate}
    \subsubsection{Resolution at acceleration level}\label{sec:Resolution at acceleration level}
    The resolution at acceleration level is given by:
    \begin{equation}
        \ddot{q} = J(q)^{\#}(\ddot{r}-\dot{J}(q)\dot{q}) + (I - J(q)^{\#}J(q))\ddot{q}_0
    \end{equation}
    where $\ddot{q}_0$ is a generic preferred joint acceleration, that can be 
    written as: $\ddot{q}_0 = - \nabla H_{range} - K_D \dot{q}$ where $K_D$ is a damping factor.
    This yields the minimum acceleration norm solution.
\subsubsection{ex: Find the joint velocity that minimizes the $H_{range}$ and scale it if not feasible}
The steps are the following:
\begin{enumerate}
    \item Use the projected gradient method to find the joint
     velocity that minimizes the $H_{range}$
    \item If the joint velocity is not feasible we want to find a 
    scaling factor k that makes it feasible. 
    We want $\dot{r}new = k \dot{r}$
    \item Since we can write the new version of the $\dot{q}$ (that is $\dot{q}_{max}$) by substituting 
    $\dot{r}$ with $\dot{r}new$ in the projected gradient method equation
    \item We extract k from the equation, and we compute the new $\dot{r}$ as $\dot{r}new = k \dot{r}$
\end{enumerate}
\subsubsection{ex: Find a solution in case of presence of task error}
The steps are the following:
\begin{enumerate}
    \item To find the acceleration that brings the 
    task error back to zero we modify the tsandard equation by 
    adding the PD terms:
    $\ddot{q} = -K_v\dot{q} + J(q)^{\#}(\ddot{r} + K_D(\dot{r}_d - J_r(q)\dot{q}) + K_P(r_d - p_x(q))) - (I - J(q)^{\#}J(q))\ddot{q}_0 + J_r(q)K_v\dot{q}$
\end{enumerate}
    \subsection{Task augmentation methods} \label{sec:Task augmentation methods}
The task augmentation methods are used to solve the redundancy problem.
\subsubsection{Task Priority}\label{sec:Task Priority}
This method gives a solution that satisfies the several task in priority order in the following way:
\begin{equation}
    \dot{q} = J_1^\# \dot{r}_1 + (I - J_1^\# J_1)v_1
\end{equation}
where $v_1$ is made to satisfy also (possibly) the lower priority task.
\begin{equation}
    v_1 = (J_2 P_1)^{\#}(\dot{r}_2 - J_2 J_1^\# \dot{r}_1)+(I -( J_2 P_1)^{\#}(J_2P_1))^{\#}v_2
\end{equation}
\subsubsection{General recursive task priority formulation}
In the general recursive task priority formulation, we have to do the following:
\begin{enumerate}
    \item We start with $\dot{q}_0 = 0$ and $P_{A0}=I$
    \item We compute the solution at k-th level as: 
    $\dot{q}_k = \dot{q}_{k-1} + (J_k P_{A,k-1})^{\#}(\dot{r}_k - J_k \dot{q}_{k-1})$
    \item We compute the projection matrix as: $P_{A,k} = P_{A,k-1} - (J_kP_{A,k})^{\#} J_k P_{A,k-1}$
\end{enumerate}
where $P_{A,k-1} = I - J_{A,k-1}^{\#} J_{A,k-1}$ and $J_{A,k-1}$ is the jacobian of the $k-1$ task.
\subsubsection{ex: Check if the tasks can be executed in parallel}
The steps are the following:
\begin{enumerate}
    \item Compute the Jacobian of the two tasks
    \item Compute the rank of the two Jacobians and the rank of the extended Jacobian
    \item If the rank of the extended Jacobian is equal to the sum of the ranks of the two Jacobians,
        the tasks can be executed in parallel
    \item If the rank of the extended Jacobian is less than the sum of the ranks of the two Jacobians,
        the tasks can not be executed in parallel
\end{enumerate}
\subsection{SNS method}
The SNS method is used to solve the redundancy 
problem with hard constraints. It saturates one overdriven joint command at a time, until a feasible and better
performing solution is found.
\begin{equation}
    \dot{q} = (JW)^{\#} s\dot{x} + (I - (JW)^{\#}J) \dot{q}_N
\end{equation}
where s is the scaling factor,  W is the diagonal 0-1 matrix
 , $\dot{q}_N$ contains
 saturated
 joint velocities.\\
The algorithm is the following:
\begin{enumerate}
    \item Initialize W matrix with 1 on the diagonal if the joint is enabled, 0 if it is not
    \item Compute the solution as: $\dot{q} = J^\# \dot{r}$
    \item Check joint velocity limits: $\dot{q}_{min} \leq \dot{q} \leq \dot{q}_{max}$
    \item Compute task scaling factor on the most critical joint
    \item If a larger scaling factor is found, update W matrix
    \item Disable the most critical joint by forcing it to its max velocity and repeat from step 2
\end{enumerate}
    \subsubsection{ex: Find the acceleration command that satisfies the limits}
We have to do the following:
\begin{enumerate}
    \item Compute the acceleration command $\ddot{q}=J(q)^{\#}(\ddot{p}-\dot{J}(q)\dot{q})$
    \item Check if the acceleration command satisfies the limits
    \item $\ddot{Q}_{min}= \max(A_{min}, \frac{V_{min}-\dot{q}}{T_c})$
    \item $\ddot{Q}_{max}= \min(A_{max}, \frac{V_{max}-\dot{q}}{T_c})$
    \item We have to check that $\ddot{Q}_{min} \leq \ddot{q} \leq \ddot{Q}_{max}$
    \item If not we do:
    \item $\ddot{p}_{new} = \ddot{p} - J_i A_i$ where i is the 
    index of the most exceeding joint $J_i$ is the i-th column of the Jacobian and $A_i$ is the i-th joint max acceleration
    \item Re-compute $\ddot{q}$ as $\ddot{q}=J(q)^{\#}_{new}(\ddot{p}_{new}-\dot{J}(q)_{new}\dot{q}_{new})$ where 
    $J(q)_{new}$ is the Jacobian with the i-th column removed and $J(q)_{new}$ is the i-th joint max acceleration
\end{enumerate}
\subsubsection{ex: Find the joint velocity that executes the task and that satisfies the joint velocities bound}
We have to do the following:
\begin{enumerate}
    \item Compute $\dot{q}=J(q)^{\#}\dot{p}$
    \item Check if the joint velocity satisfies the limits
    \item If not find  the most exceeding joint i and recompute the 
    velocity $\dot{p}$ as $\dot{p}_{new}=\dot{p} - J_i*V_i$
    where $J_i$ is the i-th column of the Jacobian and $V_i$ is 
    the i-th joint max velocity
    \item Re-compute $\dot{q}$ as $\dot{q}=[(J_1 \dots J_{i-1}J_{i+1} \dots J_n)^{\#}\dot{p}_{new}]$
\end{enumerate}

\subsubsection{ex: Find the joint velocity with minimum norm that executes the task}
    The steps are the following:
    \begin{enumerate}
        \item Compute the Jacobian J
        \item Compute the rank of J
        \item If the rank of J is full, use the pseudo-inverse
        \item Otherwise use the pseudoinverse on $J_i$ containing the J
         with only the independent rows use the corresponding rows in $\dot{r}$
    \end{enumerate}





\section{Dynamic redundancy resolution}
The dynamic redundancy resolution is used to solve the redundancy problem.
We have the Linear Quadratic problem:
\begin{equation}
    J(q) \ddot{q} = \ddot{x}= \ddot{r}-\dot{J}(q)\dot{q}
\end{equation}
typical objectives are:
\begin{itemize}
    \item \textbf{torque norm}: $H(q)=\frac{1}{2}*||\tau||^2$\\
    \textbf{soution}: $\tau= (J(q)M^{-1}(q))^{\#}(\ddot{r}-\dot{J}(q)\dot{q}+J(q)M^{-1}(q)n(q,\dot{q}))$\\
    when: good for short trajectories, otherwise it can lead to torque oscillations
    \item \textbf{(squared inverse inertia weighted) torque norm}: $H(q)=\frac{1}{2} ||\tau||^2_{ M^{-2}}$\\
    \textbf{soution}: $\tau= M(q)J^{\#}(q)(\ddot{r}-\dot{J}(q)\dot{q}+J(q)M^{-1}(q)n(q,\dot{q}))$\\
    when: to be preferred, good performance in general
    \item \textbf{(inverse inertia weighted) torque norm}: $H(q)=\frac{1}{2} ||\tau||^2_{ M^{-1}}$\\
    \textbf{soution}: $\tau= J^T(q)(J(q)M^{-1}(q)J^T(q))^{-1}*(\ddot{r}-\dot{J}(q)\dot{q}+J(q)M^{-1}(q)n(q,\dot{q}))$
\end{itemize}
\textbf{Note:} $n(q,\dot{q})$ is the sum of (coriolis, centrifugal, gravity) torques $c(q,\dot{q})$ and $g(\dot{q})$.
\subsection{Linear Quadratic problem}
To solve a linear quadratic problem we have to follow the following steps:
\begin{enumerate}
    \item Start from defining the objective function as: $H(q)=\frac{1}{2}(\ddot{q}-\ddot{q}_0)^T W (\ddot{q}-\ddot{q}_0)$
     or $H(q)=\frac{1}{2}(\dot{q}-\dot{q}_0)^T W (\dot{q}-\dot{q}_0)$ (identify your weight matrix from there)
     \item We have to set the constraints as $Jx=y$ and we have to identify 
        the value of x and y in our case (typically if x is $\ddot{q}$, y is $\ddot{r}-\dot{J}\dot{q}$
         while if x is $\dot{q}$, y is $\dot{r}$)
    \item Find $x_0$ by setting the derivative of the objective function to 0
    \item We have to compute the solution as: $x = x_0 + W^{-1}J^T(JW^{-1}J^T)^{-1}(y-Jx_0)$
\end{enumerate}
\subsubsection{Standard velocity case}
The Linear Quadratic problem is given by:
\begin{equation}
    \min{H(q)} = \frac{1}{2} (\dot{q}-\dot{q}_0)^T W (\dot{q}-\dot{q}_0)
\end{equation} s.t. \begin{equation}
    J(q) \dot{q} = \dot{r}
\end{equation}
The solution is given by:
\begin{equation}
    \dot{q} = \dot{q}_0 + J^{\#}(q)_W (\dot{r}-J(q)\dot{q}_0)
\end{equation}
which is equal to:
\begin{equation}
    \dot{q} = J^{\#}_W\dot{r}+(I - J^{\#}_WJ)\dot{q}_0
\end{equation}
\subsection{Quadratic programming}
The quadratic programming is used to solve the redundancy problem,
it is a generalization of the linear quadratic problem.
The general form of the quadratic programming is given by:
\begin{equation}
    \min \frac{1}{2}||J(q)\ddot{q}-\ddot{r}||^2 + \frac{1}{2}||\omega||^2
\end{equation} s.t. \begin{equation}
    C(q,\dot{q})\dot{q} - \omega \leq d
\end{equation}
with $\omega$ the slack variable, $C(q,\dot{q})$ the constraint 
matrix and $d$ the constraint vector.\newline
The solution is given by: $TBD$





\section{Dynamic Model}
The \textbf{dynamic model} of a robot provides 
the relationship between the generalized forces $u$ and the
assumed configurations over time $q(t)$ in form of 
a set of differential equations $\phi(q,\dot{q},\ddot{q})=u$\\
We have mainly two ways to obtain the dynamic model of a robot:
\begin{enumerate}
    \item \textbf{Euler-Lagrangian method}: To 
    obtain the dynamic model of a robot in \textbf{closed form} \\
     \textbf{best for }: Studying the dynamic properties and control schemes
    \item \textbf{Newton-Euler method}: To obtain the dynamic model of a robot in 
     a \textbf{recursive way}.\\
     \textbf{best for }: Implementation of control schemes
\end{enumerate}





\section{Euler-Lagrangian method}
\subsection{Euler Lagrange equation}
The Euler Lagrange equation is given by:
\begin{equation}
    \frac{d}{dt} \left( \frac{\partial L}{\partial \dot{q}} \right) - \frac{\partial L}{\partial q} = u
\end{equation}
where $L$ is the Lagrangian of the robot and $u$ are the non 
conservative forces.\\
The Lagrangian is given by:
\begin{equation}
    L(q,\dot{q})= T(q,\dot{q}) - U(q)
\end{equation}
where $T$ is the kinetic energy and $U$ is the potential energy.

\subsection{Dynamic equation}
The full dynamic equation is given by:
\begin{equation}
    M(q) \ddot{q} + C(q, \dot{q}) \dot{q} + G(q) = u
\end{equation}
where:
\begin{itemize}
    \item $M(q)$ is the inertia matrix
    \item $C(q, \dot{q})$ is the coriolis and centrifugal matrix
    \item $G(q)$ are the gravity matrix
    \item $u$ is the non conservative forces
    \item $\ddot{q}$ is the joint acceleration
    \item $\dot{q}$ is the joint velocity
    \item $q$ is the joint position
\end{itemize}
Alternatively we can write the dynamic equation as:
\begin{equation}
    M(q) \ddot{q} + S(q,\dot{q}) \dot{q} + G(q) = u
\end{equation}
\subsubsection{ex: Find the dynamic equation of the robot}\label{sec:find the dynamic equation of the robot}
The steps are the following:
\begin{enumerate}
    \item Start from defining the kinetic energy of the robot
     as: $T_i = \frac{1}{2} m_i \dot{r_i}^2 + \frac{1}{2} \omega_i^T I \omega_i$
     \item Compute the kinetic energy of the robot as: $T = \sum_{i=1}^{n} T_i$
     \item Compute the inertia matrix as: $M(q) = 2*\frac{\partial T}{\partial \dot{q}^2}$
     \item Find the coriolis and centrifugal matrix as in \ref{sec:Coriolis and centrifugal matrix}
     \item Find the gravity vector as in \ref{sec:Potential Energy}
     \item Compute the friction torque  
     \item Compute u as: $u = \tau - \tau_{friction}$ where $\tau$ is the commanded torque
     \item Compute the dynamic equation as: $M(q) \ddot{q} + C(q, \dot{q}) \dot{q} + G(q) =u$
\end{enumerate}
\subsection{Inertia matrix} \label{sec:Inertia matrix}
The inertia matrix is a matrix that contains the inertia of the robot, it can be derived by the 
following \textbf{foundamental relation}:
\begin{equation}
    T= \frac{1}{2} \dot{q}^T M(q) \dot{q}
\end{equation}
where $T$ is the kinetic energy of the robot. 
\subsubsection{Properties}
The properties of the inertia matrix are:
\begin{itemize}
    \item $M(q)$ is symmetric
    \item $M(q)$ is positive definite
    \item $M(q)$ is full rank
\end{itemize}
\textbf{Note:} The inertia matix is never a function of the first coordinate q1.
\subsubsection{Standard body inertia matrix}
\begin{itemize}
    \item \textbf{parallelepiped}: given sides $a$ (height), $b$ (width), $c$ (depth)
     and mass $m$, the inertia matrix is given by: \begin{equation}
        I = \frac{m}{12} \begin{bmatrix}
            b^2 + c^2 & 0 & 0 \\
            0 & a^2 + c^2 & 0 \\
            0 & 0 & a^2 + b^2
        \end{bmatrix}
    \end{equation}
    \item \textbf{empty cylinder}: given inner radius $a$ and outer radius $b$, height $c$ and mass $m$,
     the inertia matrix is given by: \begin{equation}
        I = \frac{m}{12} \begin{bmatrix}
            6(a^2 + b^2) & 0 & 0 \\
            0 & 3a^2 + 3b^2 + c^2 & 0 \\
            0 & 0 & 3a^2 + 3b^2 + c^2
        \end{bmatrix}
    \end{equation}
\end{itemize}
\textbf{fundamental kinematic relation of rigid bodies}:
 $v = v_{cm} + \omega \times r = v_{cm} + S(\omega) r$
\subsubsection{ex: Find the inertia matrix of a robot given the DH table and position of CoMs}
The steps are the following:
\begin{enumerate}
    \item The first thing we want to do is to compute the $T$ matrix for each link as in \ref{sec:Kinetic Energy}
    \item Now we can compute all the $T_i$ and the final $T$ as $T = \sum_{i=1}^{n} T_i$
    \item Finally we can obtain the inertia matrix as: $M(q) = 2*\frac{\partial T}{\partial \dot{q}^2}$
\end{enumerate}
\subsubsection{ex: Check if a matrix is an inertia matrix}
The steps are the following:
\begin{enumerate}
    \item Check that the matrix is not dependent from the first coordinate \textbf{(note, it can be dependent from the 
    difference between the first and another coordinate, but not directly from the first coordinate)}
    \item Check if the matrix is symmetric
    \item Check if the matrix is positive definite: \begin{itemize}
        \item Check if the determinant is negative, in that case the 
        matrix is not positive definite, if it is positive continue
        \item Compute the eigenvalues of the matrix
        \item Check if all the eigenvalues are positive, if any eigenvalue is negative the 
        matrix is not positive definite
    \end{itemize}
    
\end{enumerate}

\subsection{Coriolis and centrifugal matrix} \label{sec:Coriolis and centrifugal matrix}
The coriolis and centrifugal matrix is given by:\begin{equation}
    c_i(q, \dot{q}) = \dot{q}^T C_i(q) \dot{q}
\end{equation}
where $C_i(q)$ is the matrix of \textbf{Christoffel symbols} obtained as:
\begin{equation}
    C_i(q) = \frac{1}{2}  \left( \frac{\partial M_{i}}{\partial q} + (\frac{\partial M_{i}}{\partial q})^T - \frac{\partial M}{\partial q_i} \right)
\end{equation}
where $M_i$ is the i-th column of the inertia matrix.


\subsection{Kinetic Energy}\label{sec:Kinetic Energy}
The kinetic energy of a link is given by: (Konig theorem)
\begin{equation}
    T_i = \frac{1}{2} m_i\cdot^0v_{c,i}^T\cdot^0v_{c,i} + \frac{1}{2} \omega_i^T I_{c,i} \omega_i
\end{equation}
\textbf{Note:} $\omega_i$ and $I_{c,i}$ should be expressed in the same reference frame,
but the product $\omega_i^T I_{c,i} \omega_i$ is invariant w.r.t. any chosen frame\\
An alternative way to compute the kinetic energy is:
\begin{equation}
    T = \frac{1}{2} \dot{q}^T M(q) \dot{q}
\end{equation}

\subsubsection{Moving frames algorithm}
The moving frames algorithm is used to compute the kinetic energy of a robot.
The steps are the following:
\begin{enumerate}
    \item Initialize $^0v_0 = 0$ and $^0\omega_0 = 0$
    \item Define $\sigma = 0$ if the joint is revolute, $\sigma = 1$ if the joint is prismatic
    \item Compute $^i\omega_i = ^{i-1}R_i^T(q_i) \cdot[^{i-1}\omega_{i-1} + (1-\sigma)\dot{q}_i ^{i-1}z_{i-1}]$
    \item Compute \begin{equation} 
        ^i v_i = ^{i-1}R_i^T(q_i) \cdot[^{i-1}v_{i-1} + \sigma_i\dot{q}_i ^{i-1}z_{i-1} + ^{i-1}\omega_i \times ^{i-1}r_{i-1,i}]
    \end{equation}
    \begin{equation}
    =^{i-1}R_i^T(q_i) \cdot[^{i-1}v_{i-1} + \sigma_i\dot{q}_i ^{i-1}z_{i-1}] + ^i\omega_i \times ^ir_{i-1,i} 
\end{equation}
    \item Compute $^i v_{c,i} = ^i v_i + ^i\omega_i \times ^i r_{c,i}$
    \item Compute $T_i = \frac{1}{2} m_i ^i v_{c,i}^{T}v_{c,i} + \frac{1}{2}\cdot ^i\omega_i^T I_{c,i} ^i\omega_i$
\end{enumerate}
where $^i r_{c,i}$ is the distance of the CoM from frame i,
$^i r_{i-1,i}$ is the distance of the joint i from frame i-1,
$^{i-1}z_{i-1}$ is the z axis $\begin{bmatrix} 0 & 0 & 1 \end{bmatrix}^T$
\subsubsection{ex: Find the kinetic energy of a robot}
To obtain the kinetic energy of a robot we must calculate the kinetic 
energy of each link and sum them up.\\
To obtain $T_i$ we have to do the following:
\begin{enumerate}
    \item Check if the link is prismatic or revolute
    \item If it is prismatic: \begin{itemize}
        \item we just have to compute the linear velocity of the center of mass, that is 
        given by: the linear velocities of the previous link + the linear velocity of the
            center of mass of the link expressed in function of $\dot{q}_i$ 
        \item we can write the kinetic energy as: $T_i = \frac{1}{2} m_i \cdot^0v_{c,i}^T\cdot^0v_{c,i}$
    \end{itemize}
    \item If it is revolute: \begin{itemize}
        \item we have to compute the linear velocity $^0v_{c,i}$ of the center of mass, that is 
        given by $^0\dot{p_{c,i}}$
        \item we have to compute the angular velocity $\omega_i$  of the link, that 
        is given by the sum of the angular velocities of the previous 
        links and the current one (that is $\dot{q_i}$)
        \item we can write the kinetic energy as: $T_i = \frac{1}{2} m_i \cdot^0v_{c,i}^T\cdot^0v_{c,i} + \frac{1}{2} \omega_i^T I_{c,i} \omega_i$
    \end{itemize}
        \item Finally we can sum all the $T_i$ to obtain the kinetic energy of the robot
    \end{enumerate}
\subsection{Potential Energy} \label{sec:Potential Energy}
The potential energy of a link is given by:
\begin{equation}
    U_i = -m_i g^T r_{0,ci}
\end{equation}
where:
$r_{c,i}$ is the distance from the center of mass to the reference frame and \textbf{g} is 
the gravity vector (usually $[0,-9.81,0]^T$).\newline
\textbf{Note:} Generally, you'll find that the professor does not use the negative sign in the formula, that's because if $g = -9.81$, the minus sign gets canceled out and he directly skips this step.\newline
\textbf{Note:} If you are given the distance of the center of mass from the 
previous link and not the general frame of reference, you have to multiply the 
vector with the transformation matrix (homogeneus) that you can obtain from the DH table
\begin{equation}
    [r_{0,ci}, 1]^T=^0A_1 \cdot ^1A_2 \dots ^{i-1}A_i [r_{i,ci}, 1]^T
\end{equation}\\
we have that: \begin{equation}
    U= U_1+U_2+ \dots +U_n
\end{equation}
We can find the gravity vector $g(q)$ as:
\begin{equation}
    g(q) = \begin{bmatrix}
        \frac{\partial U}{\partial q_1} \\
        \vdots \\
        \frac{\partial U}{\partial q_n}
    \end{bmatrix}
\end{equation}
\textbf{Note:} the gravity term is equal to zero ($g(q)=0$) when:
\begin{itemize}
    \item We are moving horizontally (constant U)
    \item We have static balancing (distribution of masses)
    \item We have mechanical compensation (system of springs or closed kinematic chains)
\end{itemize}
    \subsubsection{ex: Define suitable relations between the link masses, lengths, and CoM position given an expected gravity term}
The steps are the following:
\begin{itemize}
    \item Express the position of the CoMs wrt the origin frame (using the DH values
    to find the transformation matrices) $^0r_{ci}=^0A_i ^ir_{ci}$
    \item Find $U_i = -m_i g r_{ci}$
    \item Write $g(q)$ = $\frac{\delta U}{\delta q_i}$
    \item Impose $g(q)$ = expected gravity term and find the 
    relations between the link masses, lengths and CoM positions
\end{itemize}
\subsubsection{ex: Define a tight bound on the norm of the square matrix $\frac{\partial{g}}{\partial{q}}$} \label{sec:define a tight bound on the norm of the square matrix}
The steps are the following:
\begin{itemize}
    \item First compute the gravity vector $g(q)$
    \item Then perform $\frac{\partial{g}}{\partial{q}}=\frac{^2\partial{U}}{\partial{q}^2}$
    \item Compute the norm as: $||\frac{\partial{g}}{\partial{q}}||= \sqrt{\lambda_{max}}$
   where $\lambda_{max}$ is the maximum eigenvalue of the matrix $(\frac{\partial{g}}{\partial{q}}^T \frac{\partial{g}}{\partial{q}})$
    \item Find the bound $\alpha$ as the highest possible value of the norm (by setting arbitrarly q values)
\end{itemize}

\subsection{Energy Conservation}
The total robot energy is expressed as: 
\begin{equation}
    E = T + U = \frac{1}{2}\dot{q}^T M \dot{q} + U(q)
\end{equation}
The evolution of the total energy over time is given by:
\begin{equation}
    \dot{E}=\dot{q}^Tu+\frac{1}{2}\dot{q}^T(\dot{M}(q)-2S(q,\dot{q}))\dot{q}
\end{equation}
If the potential energy is constant: $\dot{E}=\dot{q}^T u = \dot{L}$
\subsection{Additional dynamic terms}
\subsubsection{Friction torque}
The viscous friction torque is given by:
\begin{equation}
    u_{V,i}=-F_{V,i}\dot{q}_i
\end{equation}
The Coulomb friction torque is given by:
\begin{equation}
    u_{C,i}=-F_{C,i}sign(\dot{q}_i)
\end{equation}
\subsubsection{Electrical motors}
Motor $i$ mounted on link $i-1$ (or before) often with 
its spinning axis aligned with joint axis i.\\
The mass is added to the link $i-1$ and the inertia is
added to the robot kinetic energy.
\subsubsection{Complete dynamic model}
The complete dynamic model is given by:
\begin{equation}
    (M(q)+B_m(q))\ddot{q} + C(q,\dot{q})\dot{q} + g(q) + F_v\dot{q} + F_c sign(\dot{q}) = \tau
\end{equation}
where $B_m(q)$ is the motor inertia matrix, $F_v$ is the 
viscous friction coefficient and $F_c$ is the Coulomb friction 
coefficient and $\tau$ are motor torques after the reduction gears.\\
To introduce \textbf{elasticity} we can add a term $K(q)q$ to the
left side of the equation.
\subsection{Structural properties in dynamic models}
The coriolis term $C(q,\dot{q})$ can be  rewritten as $S(q,\dot{q})\dot{q}$ where 
S is the skew-symmetric matrix.\\
The matrix $\dot{M}(q)-2S(q,\dot{q})$ is symmetric.
\subsubsection{ex: Find matrix $S$ that satisfies $\dot{M}(q)-2S(q,\dot{q})$, checking if it's skew symmetric}
The steps are the following:
\begin{enumerate}
    \item Compute the coriolis matrix as in \ref{sec:Coriolis and centrifugal matrix}
    \item Compute the skew symmetric matrix as: $S(q,\dot{q}) = [S_1;...;S_n]$ where $S_i$ is 
    the skew symmetric matrix obtained by multiplying $S_i = dq^T \cdot C_i$
    \item Check if the matrix $\dot{M}(q)-2S(q,\dot{q})$ is skew symmetric
\end{enumerate}
\subsubsection{ex: Find a different (non standard) Skew symmetric matrix $S'$
that satisfies $\dot{M}(q)-2S'(q,\dot{q})$ and a matrix $S''$ that does not satisfy the property}
The steps are the following:
\begin{enumerate}
    \item Compute the coriolis matrix as in \ref{sec:Coriolis and centrifugal matrix}
    \item Compute the skew symmetric matrix as: $S(q,\dot{q}) = [S_1;...;S_n]$ where $S_i$ is 
    the skew symmetric matrix obtained by multiplying $S_i = dq^T \cdot C_i$
    \item Find a different matrix $S'= S + SS$ that satisfies the property by adding 
    a skew symmetric matrix SS to the original $S$ (Note: $SS*\dot{q}$=0)
    \item Find a matrix $S'' = S + SS$ that does not satisfy the property
    by adding to the original $S$ a matrix that is not skew symmetric $SS$
    but still is a feasible factorization (i.e. $S''(q,\dot{q}) \dot{q} 
    = c(q,\dot{q})$, ie $SS*\dot{q} = 0$)
\end{enumerate}





\section{Linear parametrization of the dynamic model}
Each link is characterized by 10 dynamic parameters:
\begin{itemize}
    \item 1 mass $m_i$
    \item 3 coordinates of the center of mass $r_{c,i} = \begin{bmatrix}
        r_{c,i}^x & r_{c,i}^y & r_{c,i}^z
    \end{bmatrix} ^T$
    \item 6 elements of the inertia matrix $I_{c,i} = \begin{bmatrix}
        I_{c,i}^{xx} & I_{c,i}^{xy} & I_{c,i}^{xz} & I_{c,i}^{yy} & I_{c,i}^{yz} & I_{c,i}^{zz}
    \end{bmatrix} ^T$ (the upper triangular part)
\end{itemize}
Both Potential and Kinetic energy can be expressed 
as a linear combination of the dynamic parameters $pi$:
\begin{equation}
    T_i= \frac{1}{2}m_i ^iv_i^T \cdot ^i v_i + \frac{1}{2} \omega_i^T I_{c,i} ^i\omega_i + m_i ^ir_{c,i}^T S(^iv_i)*^i\omega_i
\end{equation}
\begin{equation}:
    U_i = -m_i g_0^T r_{i}-g_0^T \cdot ^0R_i m_i ^ir_{c,i}
\end{equation}
Since the dynamic model is linear on $T$ and $U$,
the dynamic model can be expressed as:
\begin{equation}
    M(q) \ddot{q} + C(q, \dot{q}) \dot{q} + G(q) = Y_{\pi}(q,\dot{q},\ddot{q}) \pi=u
\end{equation}
where $Y_{\pi}(q,\dot{q},\ddot{q})$ is the regressor matrix that has a block
upper triangular structure.\\
\textbf{Important:} The dynamic model can be rewritten as:
\begin{equation}
    Y(q,\dot{q},\ddot{q}) a = Y_{\pi}(q,\dot{q},\ddot{q}) \pi
\end{equation}
where $a$ is the vector ($p$ x 1) of the dynamic coefficients ( ie the 
combination of the dynamic parameters that matter in the dynamic model)
\subsection{Minimal parametrization}
Determining the minimal parametrization means to find the
smallest p that allows to express the dynamic model.\\
\subsubsection{ex: Find the regressor matrix Y for a parametrized matrix M such that M=Ya}
The steps are the following:
\begin{enumerate}
    \item Obtain Y on MATLAB as \verb|Y = jacobian(M, a)|
    \item In mathematically terms we have to compute the 
    derivative of $M$ w.r.t. $a$
\end{enumerate}





\section{Newton-Euler method}
\subsection{Newton and Euler equations}
\subsubsection{Newton equation}
The Newton equation is given by:
\begin{equation}
    \sum{f}_i = m \cdot \dot{v}_c
\end{equation}
where $f_i$ are force acting on the link and $v_c$ is the
 linear velocity of the center of mass.\\
 By the principle of action and reaction we have that:
    \begin{equation}
        f_i -f_{i+1} +m_ig = m_i \dot{v}_{c,i}
    \end{equation}
    where $f_i$ is the force from link $i-1$ to $i$, $f_{i+1}$ is 
    the force applied from $i$ on $i+1$, $m_i$ is the mass of the 
    link and $g$ is the gravity vector.
    \textbf{Note:} All vectors must be expressed in the same 
    reference frame.
\subsubsection{Euler equation}
The Euler equation is given by:
\begin{equation}
    \sum{\mu}_i = I \cdot \dot{\omega} + \omega \times I \omega
\end{equation}
where $\mu_i$ are the torques acting on the link
 and $\omega$ is the angular velocity of the link and I
  is the inertia matrix.\\
  By the principle of action and reaction we have that:
    \begin{equation}
        \tau_i -\tau_{i+1} + f_i \times r_{i-1,ci} + f_{i+1} \times r_{i,ci} = I_i \dot{\omega}_i + \omega_i \times (I_i \omega_i)
    \end{equation}
    where $\tau_i$ is the torque acting on the link, $\tau_{i+1}$ is 
    the torque acting on the next link, $I_i$ is the inertia matrix of the 
    link and $\omega_i$ is the angular velocity of the link.
\subsection{Recursive Newton-Euler algorithm}
The recursive Newton-Euler algorithm is used to compute the 
dynamic model of a robot. It works as follows:
\subsubsection{Initialization}
Set:
\begin{itemize}
    \item $^0\omega_0 = 0$
    \item $^0\dot{\omega}_0=0$
    \item $^0a_0 = 0$
    \item $f_{N+1} = 0$
    \item $\tau_{N+1} = 0$
\end{itemize}
\subsubsection{Forward recursion}
\begin{itemize}
    \item Compute $^i\omega_i$ and $^i\dot{\omega}_i$ as: \begin{equation}
        ^i\omega_i = ^{i-1}R_i^T \cdot[ ^{i-1}\omega_{i-1} + \dot{q}_i ^{i-1}z_{i-1}]
    \end{equation} \begin{equation}
        ^i\dot{\omega}_i = ^{i-1}R_i^T \cdot[ ^{i-1}\dot{\omega}_{i-1} + \ddot{q}_i ^{i-1}z_{i-1}  + \dot{q}_i ^{i-1}\omega_{i-1} \times ^{i-1}z_{i-1}]
    \end{equation}
    \item Compute $^i a_i$ as: \begin{equation}
        ^i a_i = ^{i-1}R_i^T \cdot (^{i-1}a_{i-1}) + ^{i}\dot{\omega}_{i} \times ^{i}r_{i-1,i} + ^{i}\omega_{i} \times (^{i}\omega_{i} \times ^{i}r_{i-1,i})
    \end{equation}
    \item Compute $^i a_{ci}$ as: \begin{equation}
        ^i a_{ci} = ^i a_i + ^i\dot{\omega}_i \times ^i r_{c,i} + ^i\omega_i \times (^i\omega_i \times ^i r_{c,i})
    \end{equation}
\end{itemize}

\subsubsection{Backward recursion}
\begin{itemize}
    \item Compute $^i f_i$ as: \begin{equation}
        ^i f_i = ^iR_{i+1} \cdot ^{i+1}f_{i+1} + m_i \cdot (^i a_{ci} - ^ig)
    \end{equation}
    \item Compute $^i \tau_i$ as: \begin{equation}
        ^i \tau_i = ^iR_{i+1} \cdot ^{i+1}\tau_{i+1} + (^iR_{i+1}^{i+1}f_{i+1}) \times ^ir_{i,ci} -^if_i \times (^ir_{i-1,i}+^ir_{c,i})+I_i \cdot ^i\dot{\omega}_i + ^i\omega_i \times (I_i \cdot ^i\omega_i)
    \end{equation}
\end{itemize}
\subsubsection{Generalized forces}
The generalized forces are given by:
\begin{equation}
    u_i = \begin{cases}
        ^if_i\cdot ^i*z_{i-1} & \text{if the joint is prismatic} \\
        ^i\tau_i^T\cdot ^i*z_{i-1} & \text{if the joint is revolute}
    \end{cases}
\end{equation}
\subsection{Output examples}
Using the MATLAB routine $Ne_\alpha(arg1,arg2,arg3)$ we can obtain 
the following outputs:
\begin{itemize}
    \item \textbf{Complete inverse dynamics}: $Ne_{^0g}(q,dq,ddq)= M(q) \cdot ddq + C(q,dq) + g(q)$
    \item \textbf{Gravity term}: $Ne_{^0g}(q,0,0) = g(q)$
    \item \textbf{Coriolis term}: $Ne_{0}(q,dq,0) = C(q,dq)$
    \item \textbf{I-th column of the inertia matrix}: $Ne_{0}(q,0,e_i) = M_i(q)$ where $e_i$
     is the i-th column of the identity matrix
    \item \textbf{Generalized momentum}: $Ne_{0}(q,0,\dot{q}) = M(q) \cdot \dot{q}$
\end{itemize}
\textbf{Note:} For the command $Ne_{0}(q,0,0)$, if a force $F_e$
is applied to the end effector, the output is: $Ne_{0}(q,0,0) = J^T(q)F_e$
where J is the Jacobian matrix of the robot.





\section{Basic control}
There are two main types of control schemes:
\begin{itemize}
    \item \textbf{Open loop control}: the control input is computed 
    without any feedback from the system
    \item \textbf{Closed loop control}: the control input is computed 
    using feedback from the system
\end{itemize}
Types of control systems base on uncertainty:
\begin{itemize}
    \item \textbf{Feedback control}: small variations of parameters and initial
    conditions
    \item \textbf{Robust control}: large variations of parameters and
     initial conditions
     \item \textbf{Adaptive control}: improves performance online, 
     adapting the control law to unknown
     range of uncertainties
    \item \textbf{Intelligent control}: control based on experience
\end{itemize}
The different types of control are based on the type of
error and task can be observed in Figure \ref{fig:Control system}.
\begin{figure}
    \centering
    \includegraphics[width=0.9\textwidth]{control.png}
    \caption{Control system}
    \label{fig:Control system}
\end{figure}
\subsection{Control notation}
We use a new notation for position and velocity that are now expressed as:
\begin{equation}
    x = \begin{bmatrix}
        x_1 \\
        \vdots \\
        x_2
    \end{bmatrix}=
    \begin{bmatrix}
        q_1 \\
        \vdots \\
        q_n
    \end{bmatrix}
\end{equation}
\subsection{Equilibrium}
There are two types of equilibrium:
\begin{itemize}
    \item \textbf{unforced equilibrium}: $u=0$ zero velocity and 
    no gravity forces on the robot
    \item \textbf{forced equilibrium}: $u=u(x)$ zero velocity and
        $u(x_e)=g(x_{e1})$
\end{itemize}

\subsection{Stability of dynamical systems}
Stability of $x_e$ is defined as:
\begin{equation}
\forall \epsilon > 0, \exists \delta > 0 : ||x(t_0)-x_e||<\delta \Rightarrow ||x(t)-x_e||<\epsilon
\end{equation}
\subsubsection{Asymptotic stability}
Asymptotic stability of $x_e$ is defined as:
\begin{equation}
\exists \delta > 0 : ||x(t_0)-x_e||<\delta \Rightarrow \lim_{t \to \infty} ||x(t)-x_e||=0
\end{equation}
\textbf{Note:} This definition is in the sense of Lyapunov
\subsubsection{Exponential stability}
Exponential stability of $x_e$ is defined as:
\begin{equation}
    \exists \delta, c, \lambda > 0 : ||x(t_0)-x_e||<\delta \Rightarrow ||x(t)-x_e|| \leq c e^{-\lambda (t-t_0)} ||x(t_0)-x_e||
\end{equation}
\begin{itemize}
\item Can estimate the time to approximately converge
\item Typically this is a local property
\end{itemize}
\textbf{Note:} A necessary condition for $x_e$ to be both
globally asymptotically stable and exponentially stable is that
it is the only equilibrium point of the system.

\subsection{Lyapunov stability}
\subsubsection{Lyapunov candidate}\label{sec:Lyapunov candidate}
A Lyapunov candidate is a function $V(x)$ that is positive definite
that can be used to prove the stability of a system.\\
Examples of Lyapunov candidates are:
\begin{itemize}
    \item For a dynamic system: $V(x) = T(x) + U(x)$ + eventual 
    additional terms
\end{itemize}
\subsubsection{ex: Write the Lyapunov candidate given a dynamic model and a control law}
The steps are the following:
\begin{enumerate}
    \item Given the dynamic models of the system we want 
    to build the Lyapunov candidate $V$ by adding all the terms in the 
    system as follows:
    \begin{itemize} 
        \item For each mass $m_i$ we have to add the kinetic energy $\frac{1}{2}m_i q_i^2$
        \item For each spring $k_i$ we have to add the potential energy $\frac{1}{2}k_i e_i^2$
        \item For each virtual spring $K_{p,i}$ we have to add the potential energy $\frac{1}{2}K_{p,i} e_i^2$
        \item For the gravity term we have to add the potential energy $-m_i g q_i$
        \item Friction is not considered in the Lyapunov candidate
        \item Dampers are not considered in the Lyapunov candidate
    \end{itemize}
    \item Then we build the differentiated version of V, $\dot{V}$. The terms can be 
    differentiated as follows:
    \begin{itemize}
        \item For the mass term \begin{itemize}
            \item if $m_i$ is in function of q, it becomes $\frac{1}{2}\dot{q}^T\dot{m_i}\dot{q}+\dot{q}^Tm_i\ddot{q}$
            \item else it becomes $\dot{q}^Tm_i\ddot{q}$
        \end{itemize}
        \item For each spring, the term becomes $\dot{e}^Tk_i\dot{e}$
        \item For each virtual spring, the term becomes $-\dot{q}^TK_{p,i}\dot{e}$
        \item For each gravitational term, the term becomes $-m_i g \dot{q}$
    \end{itemize}
    \item \textbf{Note}: the $e$ in springs and virtul springs is usually different (in virtual spring
    is the differnece between desiredand actual value)
    \item Prove that $\dot{V}(x) \leq 0$ along the trajectories of the system. To simplify 
    the proof you can use the following property \begin{equation}
        \dot{q}^T(\dot{M}(q) - 2S(q,\dot{q}))\dot{q} = 0
    \end{equation} 
    We usually proceed in the simplifications by substituting $m_i \ddot{q}$ with
     the value from the control loop
     
\end{enumerate}
    \subsubsection{Lyapunov stability}
Lyapunov stability of $x_e$ is defined as:\newline
\textbf{Sufficient condition of stability}
\begin{equation}
    \exists V \text{candidate} : \dot{V}(x) \leq 0 \text{along the trajectories of the system}
\end{equation}
\textbf{Sufficient condition of asymptotic stability}
\begin{equation}
    \exists V \text{candidate} : \dot{V}(x) < 0 \text{along the trajectories of the system}
\end{equation}
\textbf{Sufficient condition of instability}
\begin{equation}
    \exists V \text{candidate} : \dot{V}(x) > 0 \text{along the trajectories of the system}
\end{equation}

\subsection{LaSalle Theorem}\label{sec:LaSalle Theorem}
La Salle's theorem states that if $\exists V \text{candidate}$ such 
that: $\dot{V}(x) \leq 0$ along the trajectories of the system, then:\\
system trajectories are bounded and converge to the largest invariant
set \\$M \subseteq S = \{x \in \mathbb{R}^n | \dot{V}(x) = 0\}$\\
\textbf{Corolloary:} $M \equiv {x_e} \rightarrow$ asymptotic stability

\subsection{Stability of dynamical systems}
For general time-varying systems, the stability can be analyzed using the Lyapunov method.
\subsubsection{Barbalat Lemma}
Barbalat's lemma states that if:
\begin{itemize}
    \item a function V(x,t) is lower bounded
    \item $\dot{V}(x,t) \leq 0$ 
\end{itemize}
then $\lim_{t \to \infty} \dot{V}(x,t) = 0$.\\
\textbf{Corollary:} if a Lyapunov candidate V(x,t) 
satisfies Barbalat's lemma, then the conclusion of 
LaSalle's theorem holds (ie. the system is asymptotically stable)

\subsubsection{ex: Verify the unique asymptotic stability of a closed loop
system in a given equilibrium}
Ways to verify the asymptotic stability of a
closed loop system:
\begin{itemize}
    \item \textbf{Lyapunov + LaSalle}: We have to find a Lyapunov
    candidate V(x) \ref{sec:Lyapunov candidate} and prove that $\dot{V}(x) < 0$ and then apply
    LaSalle's theorem \ref{sec:LaSalle Theorem}
    \item \textbf{Linearization}: Analyzes the linearized version of the system
    dynamics, obtained by a first-order Taylor expansion around the desired closed-loop equilibrium
    point
\end{itemize}
\subparagraph{Lyapunov + LaSalle}:\\
\label{sub:Lyapunov + LaSalle}
The steps are the following:
\begin{enumerate}
    \item Write (if not given) the dynamic model of the system 
    \begin{itemize}
        \item For system of masses: $\Rightarrow$ \ref{sec:ex: find the dynaic model of a system of masses}
        \item For a robot: $\Rightarrow$ \ref{sec:find the dynamic equation of the robot}
    \end{itemize}
    \item Write the control law (if not given). (Note: if you have multiple
    dynamic equations there might be some conditions on the control law, depending
    on the relation between the dynamic equations)
    \item Obtain the closed loop system equations (by 
    equating the right side of the dynamic model(s) to the control law)
    \item Find a Lyapunov candidate V(x) \ref{sec:Lyapunov candidate}. 
    Note: one single Lyapunov candidate independently of 
    the number of dynamic equations
    \item Verify that $V(x)$ is positive for all states and for 
    the desired state it is equal to zero (when $x = x_d$)
    \item Compute $\dot{V}(x)$
    \item Prove that $\dot{V}(x) < 0$
    \item Prove that the equilibrium is the only one by applying
     LaSalle's theorem \ref{sec:LaSalle Theorem} (check that 
     $\dot{V}(x) = 0$ only at the equilibrium)
    \item Impose $\ddot{x}=0$ and prove that this happens only for 
    $x=x_d$
\end{enumerate}
\textbf{Linearization}:\\
The steps are the following:
\begin{enumerate}
    \item Write (if not given) the dynamic model of the system 
    \begin{itemize}
        \item For system of masses: $\Rightarrow$ \ref{sec:ex: find the dynaic model of a system of masses}
        \item For a robot: $\Rightarrow$ \ref{sec:find the dynamic equation of the robot}
    \end{itemize}
    \item Bring to Laplace domain the dynamic model
    \item Prove asymptotic stability of the linearized system
    with Routh criterion 
\end{enumerate}
\subsubsection{ex: Find the unique equilibrium state for the closed-loop system under a given control}
The steps are the following:
\begin{enumerate}
    \item Write (if not given) the dynamic model of the system 
    \begin{itemize}
        \item for system of masses: $\Rightarrow$ \ref{sec:ex: find the dynaic model of a system of masses}
        \item for a robot: $\Rightarrow$ \ref{sec:find the dynamic equation of the robot}
    \end{itemize}
    \item Write the control law (if not given). (Note: if you have multiple
    dynamic equations there might be some conditions on the control law, depending
    on the relation between the dynamic equations)
    \item Obtain the closed loop system equations (by 
    equating the right side of the dynamic model(s) to the control law)
    \item Substitute in the closed loop system equations the
    values of the equlibrium state (the known one, hence $\dot{q}=0$ and 
    $\ddot{q}=0$)
    \item Solve the system of equations to find the equilibrium state $q_d$
    \item Show global asymptotic stability from step 4 as in \ref{sub:Lyapunov + LaSalle} (with Lyapunov + LaSalle steps)
\end{enumerate}





\section{Which control law?}
If asked to design a control law without direct request of a specific method, you can use the following criteria
to choose the correct one:
\begin{itemize}
\item To achieve \textbf{global exponential stabilization} of 
a state with \textbf{decoupled transient evolutions
 of the position error} $\rightarrow$ y feedback linearization control in the joint space
 \item To achieve \textbf{global asymptotic stabilization} of a state, \textbf{without knowledge of the robot
 inertia matrix} $\rightarrow$ a PD control with gravity
 cancellation or iterative gravity cancellation
 \item To achieve \textbf{exponential stabilization of the end-effector
  position} $p=p_d$ up
 to kinematic singularities $\rightarrow$ feedback linearization
  control in the Cartesian space
  \item \textbf{Global asymptotic stabilization} of a state for an 
  \textbf{underactuated robot} $\rightarrow$ an approximate
   feedback linearization control
\end{itemize}





\section{PD control}
The \textbf{PD control} is a control scheme that uses the
proportional and derivative terms of the error to compute the
control input.\\
The goal is asymptotic stability of the closed-loop system:\\
$q=q_d$ and $\dot{q}=0$\\
The \textbf{Control law} is given by:
\begin{equation}
    u = K_p(q_d-q) + K_d(-\dot{q})
\end{equation}
where $K_p$ and $K_d$ are the proportional and derivative gains.\\
\subsubsection{Th.1}
In the absence of gravity the robot state $(q_d,0)$ under the given 
PD joint control law is globally asymptotically stable.
\subsection{PD control with gravity compensation}
$g(q)$ in a robot is the gravity term, and is bounded as follows:
\begin{equation}
    ||\frac{\partial g}{\partial q}|| \leq \alpha
\end{equation}
\textbf{Note:} gravity term is bounded only if:
\begin{itemize}
    \item The robot has all revolute joints
    \item The robot has both types of joints but no prismatic variable in $g(q)$
    \item All prismatic joints have limited ranges
\end{itemize}
consequently:
\begin{equation}
    ||g(q)-g(q_d)|| \leq \alpha ||q-q_d||
\end{equation}
The \textbf{Control law} is given by:
\begin{equation}
    u = K_p(q_d-q) - K_d\dot{q} + g(q_d)
\end{equation}
\textbf{Note:} If a viscous friction $-F_v\dot{q}$ is present,
 the term $-K_d\dot{q}$ can be omitted.\\
\subsubsection{Th.2}
If $K_{P,m} > \alpha$ the state $(q_d, 0)$ under the given
 PD joint control law with gravity compensation is
  globally asymptotically stable.
\subsection{Approximate gravity compensation}
The approximate gravity compensation is given by:
\begin{equation}
    u = K_p(q_d-q) - K_d(\dot{q}) + \hat{g}(q)
\end{equation}
where $\hat{g}(q)$ is an approximation of the gravity term.
It leeds to a closed loop equilibriuum whose uniqueness 
is not guaranteed.
\subsubsection{ex: find the minimum values of the gains for
 a PD control with gravity compensation}
The steps are the following:
\begin{enumerate}
    \item Find the gravity term $g(q)$ for the robot 
    \item Find  the bound $\alpha$ as in \ref{sec:define a tight bound on the norm of the square matrix}
    \item Impose $K_{P,m} > \alpha$
    \item If we have viscous friction, we can impose $K_{D,m} = 0$
\end{enumerate}
\textbf{Note:} The $Th.2$ just gives a sufficient condition, not a
     necessary one, we can observe the structure of the gravity term 
     $g(q)$ and for each entry $i$, we can say that it is sufficient that 
        $K_{P,i} > 0$ if the $i-th$ entry of the gravity term is: 
        \begin{itemize}
        \item Constant 
        \item Equal to zero
        \end{itemize}





\section{PID control}
The \textbf{PID control} is a control scheme that uses the
proportional, integral and derivative terms of the error to compute the
control input (adds the integral term to the PD control to eliminate
the steady-state error).\\
PID can be used to recover a position error due to absent or incomplete 
gravity compensation.\\
\textbf{It is independent from the robot dynamics}\\
The \textbf{Control law} is given by:
\begin{equation}
    u = K_p(q_d-q) - K_d(\dot{q}) + K_i \int_0^t (q_d-q_\tau)d\tau 
\end{equation}
\subsection{Saturated PID control}
Global asymptotic stability is not guaranteed for PID control, but can be
proven under lower bound assumption on the gains for a 
non-linear PID law:
\begin{equation}
    u = K_p(q_d-q) - K_d(\dot{q}) + K_i \int_0^t \phi(q_d-q_\tau)d\tau
\end{equation}
where $\phi$ is a saturation function.\\
Examples of saturation functions are:
\begin{itemize}
    \item $\phi(x) = \begin{cases}
        \sin x & \text{if } |x| \leq \pi/2\\
        1 & \text{if } x > \pi/2\\
        -1 & \text{if } x < -\pi/2
    \end{cases}$
    \item $\phi(x) =\tanh x = \frac{e^x-e^{-x}}{e^x+e^{-x}}$
\end{itemize}





\section{Iterative Learning for gravity compensation}
We can use the iterative method when:
\begin{itemize}
    \item The robot has a complex (or unknown) dynamic model
    \item We don't need high position gain 
    \item We don't want complex conditions on the gains
\end{itemize}
It can be used to derive sufficient conditions for global
convergence with zero final error.\\
The \textbf{Control law} at the $i$-th iteration is given by:
\begin{equation}
    u_i = \gamma K_p(q_d-q) - K_d(\dot{q}) + u_{i-1}
\end{equation}
where $\gamma$ is the learning gain, $u_{i-1}$ is 
the feedforward term, $u_0=0$ is the easiest 
initialization of the feedforward term.\\
At steady state we have:
\begin{equation}
    g(q_i)=\gamma K_p(q_d-q_i) + u_{i-1} = u_i
\end{equation}

\subsubsection{Th.3}
\label{sec:Th.3}
If:
\begin{itemize}
    \item $\lambda_{min}(K_p) > \alpha$
    \item $\gamma \geq 2$
\end{itemize}
then we are guaranteed that the sequence ${q_0,..,q_i}$ converges
 to $q_d$ (and $\dot{q}=0$)  for any initial value (global convergence).\\
\textbf{Note}: The $Th.3$ just gives a sufficient condition, not a necessary one.





\section{Trajectory tracking control}
\subsection{Inverse Dynamic control}
Given a dynamic model of a robot $M(q)\ddot{q} + n(q,\dot{q}) = u$,
(where n includes also friction and other disturbances) and a 
twice differentiable desired trajectory $q_d(t)$, we can obtain the desired 
control input $u_d$ as: \begin{equation}
    u_d = M(q_d)\ddot{q}_d + n(q_d,\dot{q}_d)
\end{equation}
If there are disturbances we need to use a feedback 
to make the control scheme more robust.\\
Possible implementations: 
\begin{itemize}
    \item \textbf{Offline}: Open loop
    \item \textbf{Online}: Closed loop
\end{itemize}
There are 2 steps in control design:
\begin{enumerate}
    \item Compensation (feedforward) or 
    cancellation (feedback) of nonlinearities
    \item Synthesis of a linear control law stabilizing
    the trajectory error to zero
\end{enumerate}
\subsection{Trajectory controllers}
Possible trajectory controllers are:
\begin{itemize}
    \item Inverse dynamic compensation (FFW) + PD control 
    \begin{equation}
        u = \hat{u}_d+ K_P(q_d-q)+K_D(\dot{q}_d-\dot{q})
    \end{equation}

    \item Inverse dynamic compensation (FFW) + variable PD
    \begin{equation}
        u = \hat{u}_d+ \hat{M}(q_d)[K_P(q_d-q)+K_D(\dot{q}_d-\dot{q})]
    \end{equation}
    or
    \item Feedback linearization + PD + FFW = \textbf{computed torque}
    \begin{equation}
        u = \hat{M}(q)[\ddot{q}+K_P(q_d-q)+K_D(\dot{q}_d-\dot{q})] + \hat{n}(q,\dot{q})
    \end{equation}

    \item Feedback linearization + PID + FFW
    \begin{equation}
        u = \hat{M}(q)[\ddot{q}+K_P(q_d-q)+K_D(\dot{q}_d-\dot{q})+K_I\int_0^t(q_d-q_\tau)d\tau] + \hat{n}(q,\dot{q})
    \end{equation}
\end{itemize}
\textbf{Note:} The global stabilization for the last two laws is obtained with
$K_P > 0$ and $K_D > 0 $

\subsubsection{Feedback linearization}
In nominal conditions ($\hat{M}=M$ and $\hat{n}=n$) we have:
\begin{equation}
    u = M(q)\ddot{q}+ n(q,\dot{q}) = M(q)a + n(q,\dot{q})
\end{equation}
So the global asymptotic stabilization of tracking error is:
\begin{equation}
    a = \ddot{q}_d + K_P(q_d-q) + K_D(\dot{q}_d-\dot{q})
\end{equation}
Under feedback linearization control, the robot 
has a dynamic behavior that is
invariant, linear and decoupled in its whole state space
($\forall(q,\dot{q})$).\\
\textbf{Linearity:} error tends to 0 exponentially.\\
\textbf{Decoupling:} each joint coordinate evolves independently
from the others.
\subsubsection{Remarks}
\begin{itemize}
    \item Desired robot trajectory can be generated from 
    Cartesian data $p_d(0), \dot{p}_d(0), \ddot{p}_d(0)$
    \item Real-time computation with Newton-Euler algorithm
\end{itemize}
\subsection{Alternative global trajectory controller}
\begin{equation}
u = M(q)\ddot{q} + S(q,\dot{q})\dot{q} + g(q) + F_V\dot{q} _d + K_Pe + K_D\dot{e}
\end{equation}
This formula above gives a global asymptotic stabilization of the tracking error,
no cancellation of nonlinearities and no linear and decoupled behaviour.
\subsection{Regulation special case}
The regulation is a special case of trajectory tracking where $q_d$ 
is constant.\\
The feedback linearization control law is given by:
\begin{equation}
    u = M(q)[K_P(q_d-q) - K_D(\dot{q})] + c(q,\dot{q}) + g(q)
\end{equation}
This is a solution to the regulation problem with exponential stability.\\
The alternative global trajectory controller is given by:
\begin{equation}
    u = K_P(q_d-q) - K_D(\dot{q}) + g(q) 
\end{equation}
(It's a simple PD control with gravity compensation)

\subsection{Repetitive motion tasks}
The repetitive motion tasks are tasks that are executed
repeatedly in a finite time.\\
The robot is reinitialized at the initial state at the beginning
of each trial.\\
The control law is made of a non-model based part (often, a
decentralized PD law) + a time-varying feedforward which is
updated before every trial\\
The update of the feedforward term is done by:
\begin{equation}
    v_{k+1}(s) = \alpha (s) u'_k(s) + \beta (s) v_k(s)
\end{equation}

If a contraction condition can be enforced 
\begin{equation}
    ||\beta(s)-\alpha(s)W(s)|| < 1
\end{equation}

then the control law is globally asymptotically stable.\\
\textbf{Note:} If the choice $\beta = 1$ satisfies the contraction
condition, then we coverge with 0 tracking error. If $\alpha(s)=1/W(s)$
then we converge in 1 iteration.
\subsection{Application to robots}
Recursive robot equation:
\begin{equation}
    u = [B_m+M(q)]\ddot{q} + [F_v+S(q,\dot{q})]\dot{q} + g(q)
\end{equation}
Control law (pre-learning):
\begin{equation}
    u = K_P(q_d-q) + K_D(\dot{q}_d-\dot{q}) + \hat{g}(q)
\end{equation}
Learning filters:
\begin{equation}
    W_i=\frac{q_i(s)}{q_{di}(s)}=\frac{K_{Pi}+K_{Di}s}{\hat{B}_{mi}s^2+(\hat{F}_{vi}+K_{Di})s+K_{Pi}}
\end{equation}
Initializaton of feedforward term (using best estimate):
\begin{equation}
    v_1=[\hat{B}_m+\hat{M}(q)]\ddot{q}_d + [\hat{F}_v+\hat{S}(q,\dot{q})]\dot{q}_d + \hat{g}(q)
\end{equation}
\subsubsection{Control with approximate feedback linearization}
Dynamic model, its nominal part and (unstructured) uncertainty:
\begin{equation}
    M(q)\ddot{q} + n(q,\dot{q}) = \tau 
\end{equation}
\begin{equation}
    M = \hat{M} + \Delta M 
\end{equation}
\begin{equation}
    n = \hat{n} + \Delta n
\end{equation}
Model based feedback linearization:
\begin{equation}
    \tau_{fl} = \hat{M}(q)a + \hat{n}(q,\dot{q}) 
\end{equation}
Resulting closed-loop dynamics with perturbation:
\begin{equation}
\ddot{q}=a + \delta(q,\dot{q},a)
\end{equation}
\begin{equation}
    \delta = (M^{-1}\hat{M}-I)a + M^{-1}(\hat{n}-n)
    \end{equation}
Control law for tracking $q_d(t)$ (PD with feed forward + regressor $\epsilon_k$):
\begin{equation}
    a= u_k + \epsilon_k = K_P(q_d-q) + K_D(\dot{q}_d-\dot{q}) + \ddot{q}_d + \epsilon_k
\end{equation}
The regressor term is obtained with a Gaussian Process (GP) model:
\begin{equation}
    \epsilon_k \circ \mathcal{N}(\mu(\hat{X_k},\sigma^2(\hat{X_k})))
\end{equation}
\subsubsection{ex: Achieve global exponential stabilization with 
decoupled transient evolution of positions errors e(q,t)}
The steps are the following:
\begin{itemize}
\item Write the feedback linearization control law as: \begin{equation}
u=M[-K_D\dot{q}+K_P(q_d-q)]+c(q,\dot{q})+g(q)
\end{equation}
\item Obtain the desired error transient by choosing suitable gains 
$K_D$ and $K_P$ that satisfy the following equation: \begin{equation}
    \ddot{e}_i+K_{D,i}\dot{e}_i+K_{P,i}e_i=0
\end{equation}
\item Obtain $\dot{e}$ and $\ddot{e}$ from the given $e$ equation by differentiation 
and substitute them in the previous formula.
\item Solve by principle of polynomial identity for $K_D$s and $K_P$s
\end{itemize}

\subsubsection{ex: Solve the inverse dynamics for a desired trajectory $q_d(t)$}
The steps are the following:
\begin{enumerate}
    \item Write the dynamic model of the robot
    \item Substitute the desired trajectory in the dynamic model
    \item Solve the equation for $\tau$
\end{enumerate}
Note: If you have more than one dynamic equation, you have to
substitute the desired trajectory in all the equations,
obtain the values of the other parameter (say $\theta$) and its derivatives
in terms of the desired trajectory and then substitute them in the
equation for $\tau$.





\section{Adaptive control}
We need adaptation in robot motion when we have:
\begin{itemize}
    \item Poor knowledge of inertial payload
    \item Large uncertainties in robot dynamics parameters (
        $m_i, I_i, r_{ci}$)
\end{itemize}
We assume that DH parameters are known (including link lengths).\\
Main methodologies:
\begin{itemize}
    \item \textbf{Linear parametrization} of robot dynamics 
    \item \textbf{Nonlinear control law} of dynamic type (nonlinear control
    based on feedback linearization) 
\end{itemize}
\subsection{Adaptive Control Law}
First adaptive method: on-line modification with reference velocity.\\
$\dot{q}_r = \dot{q}_d + \Lambda(q_d - q)$ where typically $\Lambda=K_D^{-1}K_P$.\\
\textbf{Adaptive control law}:
\begin{equation}
    u = M(q)\ddot{q}_r + S(q,\dot{q})\dot{q} + g(q) + F_V\dot{q}_r + K_P(q_d-q) + K_D(\dot{q}_d-\dot{q})
\end{equation}
with adaptation law:
\begin{equation}
    \dot{\hat{a}} = \Gamma Y^T(q,\dot{q},\dot{q}_r,\ddot{q}_r)(\dot{q}_r-\dot{q})
\end{equation}
\subsection{Th.4}
The introduced adaptive controller makes the tracking error along
the desired trajectory globally asymptotically stable.
\subsubsection{Properties}
If the desired trajectory $q_d(t)$ is persistently exciting, then
the estimates of the dynamic parameters converge to the true values.\\
Condition of persistent excitation:
\begin{itemize}
\item For linear systems: The number of frequency components in the desired trajectory
should be at least twice as large as the number of unknown coefficients
\item For nonlinear systems: The condition can be checked only a posteriori
(a squared motion integral should always be positive bounded from below)
\end{itemize}

\subsection{Adaptive regulation}\label{sec:Adaptive regulation}
Adaptation in case $q_d$ is constant:
\begin{itemize}
    \item The adaptive control law remains the same:
    \begin{equation}
        u = \hat{M}(q)\ddot{q}_r + \hat{S}(q,\dot{q})\dot{q}_r + \hat{g}(q) + \hat{F}_V\dot{q}_r + K_P(q_d-q) + K_D(\dot{q}_d-\dot{q})
    \end{equation}
    and adaptation law:
    \begin{equation}
        \dot{\hat{a}} = \Gamma Y^T (q,\dot{q},\dot{q}_r,\ddot{q})(\dot{q}_r-\dot{q})
    \end{equation}
    \item If we have an adaptive version of the trajectory tracking 
    controller instead it would only need the adaptation on 
    the gravity term.
    \begin{equation}
        u = \hat{M}(q)\ddot{q}_d + \hat{S}(q,\dot{q})\dot{q}_d + \hat{g}(q) + \hat{F}_V\dot{q}_d + K_P(q_d-q) + K_D(\dot{q}_d-\dot{q})
    \end{equation}
    and adaptation law:
    \begin{equation}
        g(q) = G(q)a_g
    \end{equation}
    \begin{equation}\label{eq:adaptive regulation gravity}
        u = G(q)\hat{a}_g + K_P(q_d-q) + K_D(\dot{q})
    \end{equation}
    \begin{equation}
        \dot{\hat{a}}_g= \gamma G^T(q)(\frac{2e}{1+2||e||^2}-\beta \dot{q})
    \end{equation}
    where $e = q_d - q$ and $\gamma$ and $\beta$ are positive constants
\end{itemize}
\subsubsection{ex: How many states has the adaptive dynamic controller?}
The number of states is the number of coefficients 
in $a$. Note that if you have also a separated 
gravity term, you have to consider also the
coefficients of the gravity term (number of states = $ |a_m| + |a_g|$)
\subsubsection{ex: Find the adaptive control law that obtains global 
asymptotic tracking of a trajectory $q_d(t)$}
The steps are the following:
\begin{enumerate}
    \item Write the dynamic model of the robot and 
    obtain the linear parametrization Y
    \item Write the control law and adaptation law as in \ref{sec:Adaptive regulation}
    (removing eventually not-necessary terms)
    \item Write the formula for $q_r$ as $\dot{q}_r = \dot{q}_d + \Lambda(q_d - q)$
    where $\Lambda=K_D^{-1}K_P$
\end{enumerate}
\subsubsection{ex: Find the reduced states adaptive control law that 
obtains global 
asymptotic tracking of a trajectory $q_d(t)$ assuming 
dynamic coefficients $a_m$ are known}
The steps are the following:
\begin{enumerate}
    \item Write the gravity term as $g(q) = G(q)a_g$
    \item Write the control law and adaptation law as in \ref{eq:adaptive regulation gravity}
    \item Write the formula for $q_r$ as $\dot{q}_r = \dot{q}_d + \Lambda(q_d - q)$
    where $\Lambda=K_D^{-1}K_P$
\end{enumerate}





\section{Cartesian control}
We use the Cartesian control when the goal is the 
asymptotic stabilization of the end-effector position:\\
$p=p_d, \dot{q}=\dot{q}_d=0 \leftrightarrow \dot{p}=\dot{p}_d=0$\\
Note:\\
if $m= n$, then $\dot{q} \implies \dot{p} = 0$\\
if $m<n$ then the goal is not uniquely associated
to a complete robot state.\\
The \textbf{Cartesian regulation control law} is given by:
\begin{equation}\label{eq:Cartesian regulation control law}
    u = J^T(q)K_P(p_d-p) - K_D\dot{q}+g(q)
\end{equation}
\subsection{Th.5}
Under the control law in \ref{eq:Cartesian regulation control law}, the 
robot state will converge asymptotically to the set 
$A = \{\dot{q}=0,g:K_p(p_d-f(q)) \in N(J^T(q))\}$
\subsection{Corollary}
For a given initial state $(q(0),\dot{q}(0))$, if the robot does not
encounter any singularity of $J^T(q)$ during its motion, then there 
is asymptotic
stabilization to one single state or to a set of
states such that $e_p=0, \dot{q}=0$
\subsection{Cartesian control variant}
Control law (PD control plus gravity cancellation in joint space):
\begin{equation}
    u = J^T(q)[K_P(p_d-p) - K_D\dot{p}] + g(q)
\end{equation}
\subsection{Feedback linearization in Cartesian space}
Algorithm: differentiate the output(s) as many times as needed
up to the appearance of (at least one of) the input torque(s) u ,
then verify if it is possible to solve for the input.
\begin{itemize}
\item $y=f(q)$
\item $\dot{y}=J(q)\dot{q}$
\item $\ddot{y}=J(q)\ddot{q}+ \dot{J}(q)\dot{q}= J(q)M(q)^{-1}(u-c(q,\dot{q})-g(q))+\dot{J}(q)\dot{q}$
\end{itemize}
where $\ddot{q}$ is taken from the dynamic model
\subsubsection{Th.6}
Theorem for a non-redundant robot, it is possible to exactly linearize and
decouple the dynamic behavior at the Cartesian level if and only if
$\det(J(q)) \neq 0$ 
\subsubsection{Linearization in right coordinates}
From the control law,
\begin{equation}
    u = M(q)J^{-1}(q)a + c(q,\dot{q}) + g(q) - M(q)J^-1(q)\dot{J}(q)\dot{q}
\end{equation}
we can derive:
\begin{equation}
    \ddot{y}=\ddot{p}= J(q)M^{-1}(u-c(q,\dot{q})-g(q))+\dot{J}(q)\dot{q}=a
\end{equation}
From the closed-loop equation in joint space we obtain:
\begin{equation}
    \ddot{q}=J^{-1}(q)a-J^{-1}(q)\dot{J}(q)\dot{q}
\end{equation}
Note: the design of a Cartesian trajectory tracking control is completed
by setting:
\begin{equation}
    a_i=\ddot{p}_{di}+K_{Di}(\dot{p}_{di}-\dot{p}_i)+K_{Pi}(p_{di}-p_i)
\end{equation}
Note: For $p_d$ constant, the control law is:
\begin{equation}
    u = M(q)J^{-1}(q)[K_Pe_P-K_DJ(q)\dot{q}]+c(q,\dot{q})+g(q) - M(q)J^-1(q)\dot{J}(q)\dot{q}
\end{equation}

\subsubsection{ex: Find a robot configuration q and consequent p (that is not 
$p_d$) such that the robot will not move when at rest in q}
The steps are the following:
\begin{enumerate}
    \item Write the Cartesian control law as in \ref{eq:Cartesian regulation control law}
    \item Find the singularities of the Jacobian
    \item Substitute the values of the singularities in the forward
    kinematics to find the corresponding end effector position
    \item Substitute the value of q, $\dot{q}=0$ in the control law 
    and obtain its reduced form
    \item Equate the reduced form to the model dynamics 
    (where you substitute q and $\dot{q}$ and prove that $\ddot{p}=0$
\end{enumerate}





\section{Robot interaction with the environment}
\subsection{Which type of control law?}
If asked to design a control law without direct request of a
 specific method, you can use the following criteria:
\begin{itemize}
    \item If you have knowledge of the specific geometry of the
    environment and the robot is in contact with it, you can use
    a hybrid force/velocity control
    \item If you have no knowledge of the specific geometry of the
    environment and the robot is in contact with it, you can use
    impedance control
\end{itemize}
\subsection{Robot compliance}
Robot compliance can be:
\begin{itemize}
    \item \textbf{Passive}: the robot is compliant by structure
    \item \textbf{Active}: the robot is able to control the compliance
    with \begin{itemize}
    \item Admittance control
    \item Impedance control
    \item Stiffness/compliance control
    \item Hybrid control
    \end{itemize}
\end{itemize}
in all cases for physical interaction tasks, the desired motion
specification and execution should be integrated with
complementary data for the desired force.
\subsection{Ideally constrained contact}
If geometry is known hybrid force/velocity control is the best choice.
In the ideal case, with a infinitely stiff environment,
we can assume $f_e = - f_r$ where $f_e$ is the force exerted by the
environment and $f_r$ is the force exerted by the robot (normally w.r.t environment).\\

\subsection{Constrained robot dynamics}
Suppose that the task variables are subject to $m < n$ (bilateral)
geometric constraints in the general $k(r)=0$ and define \begin{equation}
h(q)=k(f(q))=0
\end{equation}
the constrained robot dynamics can be derived using again the
Lagrange formalism, by defining an augmented Lagrangian as:
\begin{equation}
    L(q,\dot{q},\lambda)=T(q,\dot{q})-U(q)-\lambda^T h(q)
\end{equation}
where the Lagrange multipliers $\lambda$ (a $m$-dimensional vector) can
be interpreted as the generalized forces that arise at the contact
when attempting to violate the constraints.\\
By applying the Euler-Lagrange equations we obtain 
the constrained robot dynamics as:
\begin{equation}
    M(q)\ddot{q} + c(q,\dot{q}) + g(q) = u +A^T(q)\lambda
\end{equation}
where A(q) is the Jacobian of the constraints $h(q)$, in fact: $A(q)=\frac{\partial h}{\partial q}$\\
We can obtain the value of $\lambda$ by doing the following:
\begin{itemize}
\item Differentiate the constraint $h(q)=0$ with respect to time (twice),
obtain $\ddot{h}=A(q)\ddot{q}+\dot{A}(q)\dot{q}=0$
\item Substitute the value of $\ddot{q}$ from the constrained robot dynamics
\item Obtain the value of $\lambda$ by solving the equation 
\begin{equation}
    \lambda = (AM^{-1}A^T)^{-1}(AM^{-1}(c+g-u)-\dot{A}\dot{q})
\end{equation}
The final constrained dynamic model can be rewritten as:
\begin{equation}
    M(q)\ddot{q} = [I -A^T(q)(A_M^{\#}(q))^T](u-c(q,\dot{q})-g)-M(q)A_M^{\#}(q)\dot{A}(q)\dot{q}
\end{equation}
where $A_M^{\#}(q)$ is the Moore-Penrose pseudo-inverse of $A(q)$:\\
\begin{center}
    $A_M^{\#}(q)=M(q)^{-1}A^T(q)(A(q)M^{-1}(q)A^T(q))^{-1}$
\end{center}
\end{itemize}

\subsection{Reduced robot dynamics}
By imposing $m$ constraints $h(q)=0$ on the $n$ generalized coordinates $q$,
it is also possible to reduce the description of the constrained robot
dynamics to a $n-m$ dimensional configuration space.\\
Along with that, we must remember that when dealing with constraint dynamics, the general form of $h(q)$ is:

\begin{equation}
    h(q) = 
    \begin{bmatrix}
        q_1(t) - k_1 \\ 
        q_2(t) - k_2 \\ 
        q_3(t) - k_3 \\ 
        \vdots \\ 
        q_n(t) - k_n
    \end{bmatrix}
\end{equation}

but, when solving the exercises, you consider only the constrained joints.

For example if you have a constraint on the first joint, your $h(q)$ will be:
\begin{center}
    $h(q) = 
    \begin{bmatrix}
        q_1(t) - k_1 \\ 
        q_2(t) - 0 \\ 
        q_3(t) - 0 \\ 
        \vdots \\ 
        q_n(t) - 0
    \end{bmatrix}
    =
    \begin{bmatrix}
        0 \\ 
        q_2(t) \\ 
        q_3(t) \\ 
        \vdots \\ 
        q_n(t)
    \end{bmatrix}$
\end{center}

which automatically translates in simply considering $h(q) = q_1(t) - k = 0$, as the constraints for $q_{2}, q_{3}, ... , q_{n}$ are not present.

Start from a constraint matrix A(q) and select D(q) such that
\begin{equation}
\begin{bmatrix}
    A(q)\\
    D(q)
\end{bmatrix}
\end{equation}

in a non-singular matrix, then the matrices $E(q)$ and $F(q)$ can be found in this way:
\begin{equation}
    \begin{bmatrix}
        A(q)\\
        D(q)
    \end{bmatrix}^{-1}= \begin{bmatrix}
        E(q) & F(q)
    \end{bmatrix}
    \end{equation}
Then we define the vector of pseudovelocities $v$ as:
\begin{equation}
    v = D(q)\dot{q}
\end{equation}
From that we can obtain $\dot{q}$ and $\ddot{q}$ as:
\begin{equation}
    \dot{q} = F(q)v
\end{equation}
\begin{equation}
    \ddot{q} = F(q)\dot{v}  - (E(q)\dot{A}(q) + F(q)\dot{D}(q))F(q)v
\end{equation}
From that we can rewrite the constrained robot dynamics as:
\begin{equation}
   (F^TMF)\dot{v} = F^T(u-c(q,\dot{q})-g + M(E\dot{A}+F\dot{D})Fv)
\end{equation}
and the lagrangian multipliers becomes:
\begin{equation}
    \lambda = E^T(MF\dot{v}-M(E\dot{A}+F\dot{D})Fv + c + g -u)
\end{equation}

N.B. In this case, your usual $M(q)$ becomes $F^T(q)M(q)F(q)$ which is the \textit{reduced inertia matrix}.





\section{Impedance control}
The impedance control imposes a desired dynamic 
behavior to the interaction between
robot end-effector and environment specified through a
complete set of mass-spring-damper equations.\\
The dynamic model of a robot in contact is given by:
\begin{equation}
M(q)\ddot{q} + c(q,\dot{q})\dot{q} + g(q) = u + J_r^T(q)F_r
\end{equation}
where $J_r(q)$ is the analyitic jacobian (from forward kinematics)
and $F_r=T_r^{-T}(\psi)F$ .\\
\textbf{Note}: In case of constraint forces instead of contact
 forces we can rewrite the equation as:
\begin{equation}
    M(q)\ddot{q} + c(q,\dot{q})\dot{q} + g(q) = u + A^T(q)\lambda
\end{equation}
where $A(q)$ is the Jacobian of the constraints and $\lambda$ is 
the reaction force intenity due to the constraints.
\subsection{Dynamic model in Cartesian coordinates}
The dynamic model in Cartesian coordinates is given by:
\begin{equation}\label{eq:dynamic model in Cartesian coordinates}
    M_r(q)\ddot{r} + S_r(q,\dot{q})\dot{r} + g_r(q) = J_r^{-T}(q)u+F_r
\end{equation}
where:
\begin{itemize}
    \item $M_r(q)=J_r^{-T}(q)M(q)J_r^{-1}(q)$
    \item $S_r(q,\dot{q})=J_r^{-T}(q)S(q,\dot{q})J_r^{-1}(q) - M_r(q)\dot{J}_r(q)J_r^{-1}(q)$
    \item $g_r(q)=J_r^{-T}(q)g(q)$
\end{itemize}
\subsection{Design of the control law}
The design happens in two steps:
\begin{enumerate}
\item Feedback linearization in the Cartesian space
 (with force measure) \begin{equation}
 u=J_r^T(q)[M_r(q)a+ S_r(q,\dot{q})\dot{r} + g_r(q) - F_r]
 \end{equation}
\item Imposition of a dynamic impedance model \begin{equation}
M_m(\ddot{r}-\ddot{r}_d) + D_m(\dot{r}-\dot{r}_d) + K_m(r-r_d) = F_r
\end{equation}
where $M_m, D_m, K_m$ are the desired mass, damping and stiffness.\\
that is realized by choosing: \begin{equation}
a = \ddot{r}_d + M_m^{-1}[D_m(\dot{r}_d-\dot{r})+K_m(r_d-r)+F_r]
\end{equation}
\end{enumerate}
\subsubsection{Control law in joint coordinates}
\begin{equation}
    \begin{split}
u = & M(q)J_r^{-1}(q){\ddot{r}_d-\dot{J}_r(q)\dot{q}+M_m^{-1}[D_m(\dot{r}_d-r)+K_m(r_d-r)]}+ \\
&S(q,\dot{q})\dot{q}+g(q)+J_r^T(q)[M_r(q)M_m^{-1}-I]F_r
    \end{split}
\end{equation}
\textbf{Note}: We can choose the parameter to have a desired dynamic:
\begin{itemize}
    \item Large $M_m$  and small $K_m$ when we want low contact forces
    \item Small $M_m$ and large $K_m$ when we want good tracking of desired 
    motion trajectory
    \item Damping coefficients $D_m$ are used then to shape transient behaviors
\end{itemize}
\subsubsection{Simplifications}
If we assume $M_m = M_r$ the control law becomes:
\begin{equation}
u=M(q)J_r^{-1}(q){\ddot{r}_d-\dot{J}_r(q)\dot{q}}+S(q,\dot{q})\dot{q}+g(q)+J_r^T(q)[
    D_m(\dot{r}_d-\dot{r})+K_m(r_d-r)]
\end{equation}
This is a pure motion control law applied also during interaction,
but designed so as to keep limited contact forces
 at the end-effector leave.\\
\textbf{Note}: If the impedance model (now, non-linear) is still supposed to
represent a real mechanical system, then in correspondence to a desired
non-constant inertia there should be Coriolis and centrifugal terms such that
the control law becomes: \begin{equation}
u=M(q)J_r^{-1}(q){\ddot{r}_d-\dot{J}_r(q)J_r^{-1}(q)\dot{r}_d}+
S(q,\dot{q})J_r^{-1}(q)\dot{r}_d+g(q)+J_r^T(q)[
    D_m(\dot{r}_d-\dot{r})+K_m(r_d-r)]
\end{equation}
which guarantees convergence to zero tracking error wher $F_r=0$.\\
We have a further simplification if $r_d$ is constant. In this case, the control law becomes:
\begin{equation}
u=g(q)+J_r^T(q)[-D_m\dot{r}_d+K_m(r_d-r)]
\end{equation}
\subsection{General case $F_r \neq 0$}
For analysis we model the environment as an elastic element: 
\begin{equation}
F_r=K_e(r_e-r)
\end{equation}
where $\dot{r}=\ddot{r}=0$ a closed loop system equilibrium is given by:
\begin{equation}
K_m(r_d-r)+K_e(r_e-r)
\end{equation}
which has unique solution
\begin{equation}
r=(K_m+K_e)^{-1}K_mr_d+K_er_e)=r_E
\end{equation}
\subsubsection{Equivalence with RCC device}
\textbf{Theorem:}
If $(r_d-r) \simeq J_r(q)(q_d-q)$ and $g(q)=0$ and $D_m=0$\\
then the control law is equivalent to the one of a RCC device:\\
\begin{equation}
    u = J_r^T(q)K_mJ_r(q)(q_d-q)=K(q)(q_d-q)
\end{equation}

\subsubsection{ex: Check that the tip acceleration
 has the same direction of any force F 
in the plane applied to the tip of the robot}
The steps are the following:
\begin{enumerate}
    \item Write the dynamic model in Cartesian coordinates as in \ref{eq:dynamic model in Cartesian coordinates}
    \item Isolate the acceleration $\ddot{r}$ from the dynamic model
    \item If the acceleration is solely dependent on the force $F$,
     then the direction of the acceleration is the same as 
     the direction of $F$
\end{enumerate}
\subsubsection{ex: Write the relation between p an q given the
model dynamics in joint space}\begin{itemize}
\item Start from the relation $\dot{p}=J(q)\dot{q}$
\item By duality obtain $\tau_q=J^T(q)\tau_p$
\item Substitute the value of $\tau_q$ in the dynamic model,
and $\tau_p$ as $(\tau_1 \dots \tau_n)^T$ and obtain 
the value of $J^T(q)$
\item Use it to find p
\end{itemize}
\subsubsection{ex: Design an impedance control law for a robot}\label{sec:ex: design an impedance control law for a robot}
The steps are the following:
\begin{enumerate}
    \item Write the dynamic model in Cartesian coordinates
    \item Write the impedance model 
    \item Check if the appartent cartesian matrix $M_m$ (the one of the 
    impedance model) is equal to the natural inertia matrix $M_r$ 
    (this could be specified in the text or be a consequence of the absence of 
    force/torque sensors)
    \item If so, substitute $f_r$ obtained from the impedance model inside the 
    dynamic model
    \item From the dynamic model, isolate and obtain $u$ (obtaining the control law)
\end{enumerate}

\subsubsection{ex: Find the equilibrium point for impedance control}
The steps are the following:
\begin{itemize}
\item Write the impedance control law at steady-state ($\dot{r}=\ddot{r}=0$)
\item Isolate and find $r$ (equilibrium point)
\item Write the control law as in \ref{sec:ex: design an impedance control law for a robot}
and substitute the values of steady state ($\dot{r}=\ddot{r}=0$)
\item Compute $u$ with the $r$ previously obtained: $u_E$
\end{itemize}

Note: Remember that when there is an external \textit{constant} force $F_{c}$ and
you are asked to find $u_E$ or $\tau_E$, along with $x_E$, you simply find them by:
\begin{equation}
    x_E = x_d + K^{-1}F_{c}
\end{equation}
\begin{equation}
    u_E = J^T(q_E) F_{c} + g(q_E)
\end{equation}





\section{Hybrid control}
The hybrid control law is designed in ideal conditions, but now
unconstrained directions of motion and constrained force directions
are defined in a more direct way using a task frame formalism.\\
All non-ideal conditions (compliant surfaces, friction at the contact,
errors in contact surface orientation) are handled explicitly in the
control scheme by a geometric filtering of the measured quantities.
\subsection{Natural and artificial constraints}
\subsubsection{Natural constraints}
In ideal conditions we can define inside the task space:
\begin{itemize}
\item \textbf{End-effector motion} ($v,\omega$) is prohibited 
in $6-k$ directions
\item \textbf{Reaction forces/torques} ($f,\mu$) are absent 
in $k$ directions
\end{itemize}
\subsubsection{Artificial constraints}
The \textbf{artificial constraints} can be specified by:
\begin{itemize}
    \item \textbf{End-effector velocities} ($v,\omega$) 
    around $k$ directions where motion can occur 
    \item \textbf{Contact forces/torques} ($f,\mu$)
    $6-k$ directions where admissible environment reactions can occur
\end{itemize}
\subsection{Hybrid force/velocity control}
The general parametrization of hybrid task can be 
described by:\begin{enumerate}
    \item Robot-environment contact type:
    \begin{equation}
        \begin{bmatrix}
            v \\
            \omega
        \end{bmatrix}
        =T(s)\dot{s}
    \end{equation}
    \begin{equation}
        \begin{bmatrix}
            f \\
            \mu
        \end{bmatrix}
        =Y(s)\lambda
    \end{equation}
    \item Robot dynamics: \begin{equation} \label{eq:robot dynamics in hybrid control}
    M(q)\ddot{q}+S(q,\dot{q})+g(q)=u+J^T(q)[f,\mu]^T
    \end{equation}
    \item Robot kinematics:\begin{equation}
    \begin{bmatrix}
        v\\
        \omega
    \end{bmatrix}=J(q)\dot{q}
    \end{equation}
\end{enumerate}
The \textbf{control objective} is to impose a desired task evolution 
to the parameters $s$ of motion and $\lambda$ of force.\\
The control law is designed in two steps:
\begin{enumerate}
    \item Extract linearization and decoupling in the task frame by feedback
            \begin{equation}
            u=(MJ^{-1}T, -J^TY)(a_s, a_{\lambda})^T + MJ^{-1}(\dot{T}\dot{s}-\dot{J}\dot{q}) + S\dot{q} +g
            \end{equation}
    \item (Linear) design of $a_s$ and $a_{\lambda}$ so as to impose 
the desired dynamic
behavior to the errors $e_s = s_d - s$ and $e_{\lambda} = \lambda_d - \lambda$
\begin{equation}
a_s=\ddot{s}_d+ K_D(\dot{s}_d-\dot{s})+K_P(s_d-s)
\end{equation}
\begin{equation}
a_{\lambda}=\lambda_d + K_I \int(\lambda_d-\lambda)dt
\end{equation}
where values for $s$, $\dot{s}$ are extracted from measures 
of $q$ and $\dot{q}$ equating the descriptions
of the end-effector pose and velocity “from the robot side”
 (direct and
differential kinematics) $\dot{s}=T^{\#}J(q)\dot{q}$ and “from the environment side” while 
$\lambda$ is obtained from 
force/torque measures at end effector $\lambda=Y^{\#}(f,m)^T$
\end{enumerate}
\textbf{Note}: We assume $n=m$ and $J(q)$ out of singularity
\subsection{Force control via impedance model}
In a force-controlled direction of the hybrid task space, when the contact
stiffness is limited,
one may use impedance model ideas to 
explicitly control the contact force.\\
The impedance model is chosen as:\begin{equation}
m_m\ddot{x}+d_m\dot{x}+k_s(x-x_d)=f_d
\end{equation}
After a feedback linearization the command $a_x$ is designed as:
\begin{equation}
a_x =(1/m_m)[(f_d-f_s)-d_m\dot{x}]
\end{equation}
\subsubsection{Sources of inconsistency}
The main sources of inconsistency in force and velocity measurements can be:
\begin{itemize}
    \item Presence of friction at the contact 
    \item Compiance in the robust structure and/or at the contacy
    \item Uncertainty on environment geometry at the contact
\end{itemize}
\subsection{Estimation of unknown surface}
\begin{itemize}
    \item Normal = nominal direction of measured force
    \item Tangent = nominal direction of measured velocity
    \item Tangent direction and friction angle can also be extimated with 
    recursive least square method
\end{itemize}

\subsubsection{ex: Design the natural and 
artificial constraints for a hybrid control law. How many control loops? How many degres of freedom are needed?}
The steps are the following:
\begin{enumerate}
\item Design the task frame (usually, align the z-axis with the end effector)
\item Start by designing the natural constraints: \begin{itemize}
    \item Define the velocities (linear and angular) that are
        prohibited in the task space ($6-k$)
    \item Define the forces/torques that are absent
     in the task space ($k$)
\end{itemize}
\item Design the artificial constraints: \begin{itemize}
    \item Define the velocities (linear and angular) that are
        allowed in the task space ($k$)
    \item Define the forces/torques that are present
     in the task space ($6-k$)
\end{itemize}
\item The number of control loops is equal to the number of
artificial constraints (6, but specifically $k$ 
control loops on motion components and $6-k$ on force components)
\item The number of degrees of freedom needed is one per 
control loop
\end{enumerate}

\subsubsection{ex: Design a hybrid control law for a robot}
The steps are the following:
\begin{enumerate}
    \item Write the control law as in \ref{eq:robot dynamics in hybrid control}:
    $M(q)\ddot{q}+S(q,\dot{q})\dot{q}+g(q)=u+J^T(q)u_f$
    where $u_f$ is the vector of forces/torques expressed 
    in the original frame (same as the model)
    \item Compute the rotation matrix $R$ from origin frame to contact plane frame 
    \item Rewrite the dynamic model in the contact plane frame
    by multiplying the elements of the model by $R$, such that:
    $R^TM(q)R\ddot{q}+R^TS(q,\dot{q})R\dot{q}+R^Tg(q)=R^Tu + R^TJ^T(q)u_f$
    where $R^T u_f$ is the vector of forces/torques expressed in the contact plane frame
    \item Write the control loop for the force control
\end{enumerate}





\section{Visual servoing}
Visual servoing refers to the
use information acquired by vision sensors (cameras)
for feedback control of the pose/motion of a robot.\\
The first phase is the \textbf{image processing}.
The \textbf{vision system} provides set-point references to a cartesian 
motion controller.\\
Possible visual servoing schemes:
\begin{itemize}
    \item \textbf{Image-based visual servoing}: The control law
     is designed directly in the image space - 2D plane.
      The steps are: \begin{itemize}
        \item Image acquisition 
        \item Feature extraction
        \item Comparison with desired values (error)
        \item Generation of motion of camera/robot
      \end{itemize}
    \item \textbf{Position-based visual servoing}: The control 
    law is designed
    in the robot task space - in 3D pose (cartesian control law)
\end{itemize}
\subsection{Camera Model}
The camera model is given by:
\begin{itemize}
\item Set of lenses (thin lenses, pinhole, catadioptic lens)
\item Matrix of light-sensitive elements
\item Frame grabber
\end{itemize}
\subsubsection{Thin lens camera model}
Fundamental equation of a thin lens:
\begin{equation}
\frac{1}{Z}+\frac{1}{z}=\frac{1}{\lambda}
\end{equation}
where Z is the distance between the object and the lens,
 z is the distance between the image plane and the lens and
    $\lambda$ is the focal length of the lens.\\
    \begin{figure}
        \centering
        \includegraphics[width=0.5\textwidth]{lens1.png}
        \caption{thin lens camera model}
        \label{fig:thin lens camera model}
    \end{figure}
\subsubsection{Pinhole camera model}
The pinhole camera model is given by:
\begin{equation}
    u=\lambda \frac{X}{Z}
\end{equation}
\begin{equation}
    v=\lambda \frac{Y}{Z}
\end{equation}
with $P=(X,Y,Z)$ the point in the 3D 
space and $(u, v, \lambda$) the point in the image plane.\\
\begin{figure}
    \centering
    \includegraphics[width=0.5\textwidth]{lens2.png}
    \caption{pinhole camera model}
    \label{fig:pinhole camera model}
\end{figure}
\subsubsection{Interaction matrix}
Given a set of features parameters f =[$f_1,f_2,...,f_n$], we look 
for the (kinematic) differential relation between
motion imposed to the camera and motion of features on the
image plane \begin{equation}
\dot{f}=J(\cdot)[V, \Omega]^T
\end{equation}
where (V,$\Omega$) is a six dimensional linear/angular velocity vector and $J(\cdot)$ is the \textbf{interaction matrix}.\\
To compute the interaction matrix: \begin{enumerate}
    \item From the perspective equation we obtain:
    \begin{equation}
        \begin{bmatrix}
        \dot{u}\\
        \dot{v}
        \end{bmatrix}=
        \begin{bmatrix}
        \frac{\lambda}{z} & 0 & -\frac{u}{Z}\\
        0 & \frac{\lambda}{Z} & -\frac{v}{Z}
        \end{bmatrix}
        \begin{bmatrix}
            \dot{X}\\
            \dot{Y}\\
            \dot{Z}
        \end{bmatrix}
        =J_1(u,v,\lambda)\begin{bmatrix}
        \dot{X}\\
        \dot{Y}\\
        \dot{Z}
        \end{bmatrix}
    \end{equation}
    \item From kinematic relation, instead, we can obtain:
    \begin{equation}
        \begin{bmatrix}
        \dot{X}\\
        \dot{Y}\\
        \dot{Z}
        \end{bmatrix}=
        \begin{bmatrix}
            -1 & 0 & 0 & 0 & -Z & Y \\
            0 & -1 & 0 & Z & 0 & -X \\
            0 & 0 & -1 & -Y & X & 0
        \end{bmatrix}
        \begin{bmatrix}
            V \\
            \Omega
        \end{bmatrix}= J_2(X,Y,Z)\begin{bmatrix}
        V\\
        \Omega
        \end{bmatrix}
    \end{equation}
    \item Thus we obtain the interaction matrix $J_p$ as
    \begin{equation}
        \begin{bmatrix}
            \dot{u}\\\
            \dot{v}
        \end{bmatrix}=J_1J_2\begin{bmatrix}
            V\\
            \Omega
        \end{bmatrix}=\begin{bmatrix}
        -\frac{\lambda}{Z} & 0 & \frac{u}{Z} & \frac{uv}{\lambda} & -(\lambda + \frac{u^2}{\lambda}) & v\\
        0 & -\frac{\lambda}{Z} & \frac{v}{Z} & \lambda + \frac{v^2}{\lambda} & -\frac{uv}{\lambda} & -u
        \end{bmatrix} \begin{bmatrix}
            V\\
            \Omega
        \end{bmatrix}=J_p(u,v,Z)\begin{bmatrix}
        V\\
        \Omega
        \end{bmatrix}
    \end{equation}
\end{enumerate}
\textbf{Note:} When requested to obtain the interaction matrix 
coming from multiple points we have variables $u_i$, $v_i$ and $Z_i$
for each point i, $\lambda$ is the same for all points.
\subsubsection{Other interaction matrices}
Other examples of interaction matrices are:\begin{itemize}
\item Distance between two points in the image plane
\begin{equation}
    \begin{split}
        d & =\frac{1}{d}
        \begin{bmatrix}
            u_1-u_2 & v_1-v_2 & u_2- u_1 & v_2-v_1 
        \end{bmatrix}
        \begin{bmatrix}
            \dot{u}_1 \\
            \dot{v}_1 \\
            \dot{u}_2 \\
            \dot{v}_2
        \end{bmatrix}\\
        & = J_P(u_1,u_2,v_1,v_2) \begin{bmatrix}
        J_{p1}(u_1,v_1,Z_1) \\
        J_{p2}(u_2,v_2,Z_2)
        \end{bmatrix} 
        \begin{bmatrix}
        V\\
        \Omega
        \end{bmatrix}
    \end{split}
\end{equation}
\item Image moments:\begin{equation}
m_{ij}=\int \int x^i y^j dxdy
\end{equation}
\end{itemize}
\subsection{Robot Differential Kinematics}
In the \textbf{eye-in-hand} case, the motion to be imposed to 
the camera coincides with the desired end-effector 
linear/angular velocity.
\begin{equation}
\begin{bmatrix}
    V\\
    \Omega
\end{bmatrix}=J_m(q)\dot{q}=J_m(q)u
\end{equation}
where $J_m(q)$ is the geometric jacobian or the NMM jacobian of 
a manipulator.\\
We obtain: \begin{equation}
\dot{f}=J_p(f,Z)J_m(q)u=J(f,Z,q)u
\end{equation}
where $J(f,Z,q)$ is the \textbf{Image jacobian}.\\
In general we can obtain the control (for \textit{IBVS}) as:
\begin{equation}
    u=J^{\#}(\dot{f}_d+ke) + (I-J^{\#}J) u_0
\end{equation}
and this will exponentially stabilize the error to 0 (up to singularities).
\subsubsection{Control With Task Sequencing}
Approach: regulate only one (some) feature at the time, while keeping
“fixed” the others by unobservable motions in the image plane.\\
The steps are the following:
\begin{enumerate}
    \item Select the feature to be regulated
    \item Compute the control law for the selected feature \begin{equation}
    u=J^{\#}_1k_1e_1+(I-J^{\#}_1J_1)u_0
    \end{equation}
    \item Compute the control law for the other features \begin{equation}
    u=(I-J^{\#}_1J_1)J^T_2k_2e_2
    \end{equation}
\end{enumerate}





\section{Actuation Faults}
There are mainly 6 problems in the diagnosis of faults: \begin{itemize}
\item \textbf{Fault Detection:} the fault is detected
\item \textbf{Fault Isolation:} discriminate which particular fault f 
has occurred out of a (large)
class of potential ones
\item \textbf{Fault Identification:} the fault is quantified
\item \textbf{Fault Accomodation:} the control law is modified to take 
into account the fault
\item \textbf{FDI (Fault Detection and Isolation):} the fault is detected
    and isolated simultaneusly 
\item \textbf{Fault Tolerant Control:} the control law is modified to
    take into account the fault (passively or actively)
\end{itemize}
\subsection{Fault Types}
\begin{itemize}
    \item Instantaneus 
    \item Incipient
    \item Intermittent
    \item Concurrent 
\end{itemize}
The dynamic in case of fault is given by:
\begin{equation}
M(q)\ddot{q}+c(q,\dot{q})+g(q)+F_{V}\dot{q}+F_{S}sgn(\dot{q})=u - u_f
\end{equation}
where $u_f$ is the fault vector, $F_{V}$ is the viscous friction,
$F_{S}$ is the static friction.\\
Possible fault expressions:
\begin{itemize}
    \item Total fault: $u_{f,i}=u_{i}$
    \item Partial fault: $u_{f,i}=\epsilon u_{i}$
    \item Saturation fault: $u_{f,i}=u_{i}-sgn(u_{i})u_{max}$
    \item Bias fault: $u_{f,i}=b_i$
\end{itemize}
\textbf{Note}: We assume that the dynamic of the robot is known.
\subsection{Generalized Momentum}
The generalized momentum is defined as:
\begin{equation}
p=M(q)\dot{q}
\end{equation}
and its dynamic is given by:
\begin{equation}
\dot{p}=u - \alpha(q,\dot{q}) - u_f
\end{equation}
where \begin{equation}\label{eq:alpha}
\alpha_i(q,\dot{q})=-\frac{1}{2}\dot{q}^T\frac{\partial M}{\partial q_i}\dot{q}+g_i(q)+F_{V,i}\dot{q}_i+F_{S,i}sgn(\dot{q}_i)
\end{equation}

\subsection{FDI solution}
Definition of the residual vector $r$ as:
\begin{equation}
r=K[\int(u-\alpha(q,\dot{q}))dt-p]
\end{equation}
where:
\begin{itemize}
    \item $K$ is a gain matrix
    \item $\alpha(q,\dot{q})=-\frac{1}{2}\dot{q}^T\frac{\partial M}{\partial q}\dot{q}+g(q)+F_{V}\dot{q}+F_{S}sgn(\dot{q})$
    \item p is the generalized momentum $p = M(q)\dot{q}$
\end{itemize}
with perfect model knowledge the dynamics of $r$ is: \begin{equation}
\dot{r}=-Kr + Ku_f
\end{equation}
\subsubsection{Residual generator}
The residual generator is a system that generates the residual vector $r$.
We have:
\begin{equation}
\dot{\hat{p}}=u - \alpha(q,\dot{q}) + K(p- \hat(p))
\end{equation}
\begin{equation}
    r=K(\hat{p}-p)
\end{equation}
and $e_{obs}=u_f - r$

\subsubsection{ex: Design a residual function $r_i$ for a fault $u_i$}
The steps are the following:
\begin{enumerate}
    \item Compute the dynamic of the robot (or system), ex: 
    $M(q)\ddot{q}+c(q,\dot{q})\dot{q}+g(q)=u - u_f$
    where each entry $u_i$ of the vector $u_f$ is the fault
    of the $i$-th actuator (if present, 0 otherwise)
    \item Write the residual function as: \begin{equation}
    r_i=K[\int(u_i-\alpha_i(q,\dot{q})-r_i)dt-p_i]
    \end{equation}
    \item Compute $alpha_i(q,\dot{q})$ as in \ref{eq:alpha}
    \item Compute $p_i$ as: \begin{equation}
    p=M(q)\dot{q}
    \end{equation}
    \item Substitute the values in the residual function
\end{enumerate}

\subsubsection{ex: Give the
 expressions of the terms appearing in the 
 definition of the residual vector r 
that allows detection and isolation of collisions.
}
The steps are the following:
\begin{enumerate}
    \item Write the equation of the a possible 
    contact force $F_K$ acting on a link as 
    $r \simeq \tau_K=J_K^TF_K$
    \item Analyze the situation for each link:
    \begin{itemize}
        \item Compute the forward kinematics up to the end link k 
        \item Compute the jacobian $J_K$ of the end link k
        \item Find the forces that cannot be detected 
        as the null space of the jacobian $J_K$
        \item Find the forces that can be detected but 
        with wrong isolation as the forces that produce 
        at least one zero in the residual vector
    \end{itemize}
\end{enumerate}

\subsubsection{ex: Detect a collision through the energy-based
and momentum-based methods}
\subparagraph{Energy-based method}
The energy-based method, in general, fails to detect a collision when the
colliding force $F_c$ is orthogonal to the velocity $v_c$ of the contact point.

This means computing the following formula:
\begin{center}
    \begin{equation}
        v_{c}^T F_{c} = (J_{c}(q)\dot{q})^T F_{c}
    \end{equation}
\end{center}

If the formula above is equal to zero, then no collision is detected.

Note that the subscript $c$ denotes the collision point, meaning that if the exercise asks for the detection
of a collision at a specific point, then the Jacobian $J_{c}$ should be computed for that point.

If, for example, you have $n$ contact points, then you'd have to compute:
\begin{center}
    $v_{c1}^T F_{c1} = (J_{c1}(q)\dot{q})^T F_{c1}$ \\
    $v_{c2}^T F_{c2} = (J_{c2}(q)\dot{q})^T F_{c2}$ \\
    ... \\
    $v_{cn}^T F_{cn} = (J_{cn}(q)\dot{q})^T F_{cn}$
\end{center}

\subparagraph{Momentum-based method}
In the momentum-based method, the collision is detected when $F_c$ is not in the nullspace
of the transpose of the Jacobian $J_c$, yielding the following formula:
\begin{center}
    \begin{equation}
        \tau_{c} = J^T_{c}(q) F_{c} = 0
    \end{equation}
\end{center}

In other words, if you're able to find the contact force $F_c$ in the nullspace of the transpose of the Jacobian $J_c$,
then the momentum-based method will detect a collision and also correctly isolate it.

A simple example:

\begin{center}
1) $\mathcal{N}\left\{\boldsymbol{J}_{c1}^T\right\} = \mathcal{N}\left\{\left(\begin{array}{cc}
    -1 & 1 \\
    0 & 1
    \end{array}\right)\right\} = \emptyset \Rightarrow \boldsymbol{F}_{c1} = \left(\begin{array}{c}
    -1 \\
    0
    \end{array}\right) \notin \mathcal{N}\left\{\boldsymbol{J}_{c1}^T\right\}$
\end{center}

\begin{center}
    hence:\\
    $\quad \boldsymbol{\tau}_{c1} = \boldsymbol{J}_{c1}^T \boldsymbol{F}_{c1} = \left(\begin{array}{c}
        1 \\
        0
        \end{array}\right) \text{ [Nm]}$
\end{center}

\begin{center}
    2) $\mathcal{N}\left\{\boldsymbol{J}_{c2}^T\right\} = \mathcal{N}\left\{\left(\begin{array}{cc}
        -1 & 0 \\
        0 & 0
        \end{array}\right)\right\} = \left\{\left(\begin{array}{cc} 0 \\ 1 \end{array}\right), \left(\begin{array}{cc} 0 \\ -1 \end{array}\right)\right\}
        \Rightarrow \boldsymbol{F}_{c2} = \left(\begin{array}{c}
        -1 \\
        0
        \end{array}\right) \in \mathcal{N}\left\{\boldsymbol{J}_{c2}^T\right\}$
    \end{center}
    
    \begin{center}
        hence:\\
        $\quad \boldsymbol{\tau}_{c2} = \boldsymbol{J}_{c2}^T \boldsymbol{F}_{c2} = 0$
    \end{center}

In this case, only the second contact point (2) is in a collision state, as the torque $\tau_{c2}$ is equal to zero.





\section{Extra exercises}
\subsubsection{ex: Uniform time scaling}
If the motion is unfeasible and we need to apply uniform time scaling, 
the steps are the following:
\begin{enumerate}
    \item Compute $K_i = \sqrt{\frac{\tau_i - g_i}{\tau_{max_i} - g_i}}$
    \item Compute the new scaled time $t_i' = K_i t_i$
    \item Compute the new scaled torques $\tau_i(t_i') = \frac{\tau_i(t_i) - g_i(q(t_i))}{K_i^2} + g_i(q(t_i))$
\end{enumerate}
    \subsubsection{ex: Find q value that causes algorithmic singularity}
The steps are the following:
\begin{enumerate}
    \item Compute the Jacobian $J$ for each of the tasks
    \item Build the extended Jacobian as: $J_e = \begin{bmatrix}
        J_1 \\
        J_2
    \end{bmatrix}$
    \item Compute the determinant of the extended Jacobian
    \item Find the $q$ value that causes the determinant to be zero
    \item Substitute the $q$ value in the Jacobians
    \item Compute the rank of both the original jacobians and the extended one
    \item If the rank of the extended jacobian i.e. $rank(J_e) < rank(J_1) + rank(J_2)$
    is less than the sum of the ranks of 
    the original jacobians, we have an algorithmic singularity.
\end{enumerate}

\subsubsection{ex: Check if two tasks are in conflict}
The steps are the following:
\begin{enumerate}
    \item Compute the Jacobian $J$ for each of the tasks
    \item Build the extended Jacobian as: $J_e = \begin{bmatrix}
        J_1 \\
        J_2
    \end{bmatrix}$
    \item Build the extended desired velocity as: $\dot{r}_e = \begin{bmatrix}
        \dot{r}_1 \\
        \dot{r}_2
    \end{bmatrix}$
    \item If the extended velcity is not in the range of 
    the extended jacobian, the tasks are in conflict
\end{enumerate}

\subsubsection{ex: Find the minimum time of execution of a trajectory for a single link}
The steps are the following:
\begin{enumerate}
    \item Write the expression of the dynamic model 
    \item If we can identify the tkind of motion
     (e.g. bang-to-bang, bang-coast-bang,
      cubic polynomial smooth traj.) then \begin{enumerate}
    \item Write equations related to the
     motion the robot needs to perform
      (e.g. bang-to-bang, bang-coast-bang, 
      cubic polynomial smooth traj.) 
    \item Find acceleration $\ddot{q}$ based on robot 
    motion equation computed at previous step 
    \item Substitute it in the dynamic model $\rightarrow$ this
     will be the new $\tau$
    \item The new tau will be composed of two terms : a linear contribution $u_a$ due to acceleration and a sinusoidal contribution $u_g$ due to gravity. 
    \item Compute minimum time T based only on the acceleration contribution $u_a$ by inverse formulas (reason explained next line)
    \item \textbf{Note}: The faster will be the assigned trajectory (i.e, the smaller the total motion time T),
     the more will the acceleration term grow and dominate the gravitational term, which does not change being dependent only on the configuration q 
\end{enumerate}
\item else if the dynamic model is parametrized by a single acceleration and the motion to be executed has an unknown profile (e.g, not a bang-to-bang profile) \begin{enumerate}
 \item define two constants a and b s.t 
 \begin{equation}
 -U_1 <= a*\ddot{q} <= U_1
    \end{equation}
 and 
    \begin{equation}
 -U_2 <= b*\ddot{q} <= U_2
 \end{equation}
 \item find $\ddot{q}$ as a function of 
 the inequalities defined in previous step
 \item Express $\ddot{q}$ bound as 
 \begin{equation}
    \max(-U_1/a, -U_2/b) = \ddot{q-} \leq \ddot{q} \leq \ddot{q+} = \min(U_1/a, U_2/b)
 \end{equation}
 \item Check that the expressed conditions about $\ddot{q+}$ and $\ddot{q-}$
  are able to perform any desired motion transfer (e.g. rest-to-rest motion)
 \item In case a rest-to-rest motion needs to be
  enforced, impose equations from moto uniformemente 
  acceleration and find values of T and Ts
 \item For rest-to-rest motion end to be enforced, 
 we set $\ddot{q-}*Ts + \ddot{q+} *(T-Ts) = 0$
 \item For net displacement to be 
 achieved, $1/2*\ddot{q}-*(Ts)^2 - 1/2*\ddot{q}+*(T-Ts)^2 = displacement$
 \item Find the values of T and Ts
\end{enumerate}

\end{enumerate}

\subsubsection{ex: Find the dynamic model of a system of masses
spring and dampers} \label{sec:ex: find the dynaic model of a system of masses}
There are two ways to solve this problem:
\begin{itemize}
    \item \textbf{Lagrangian method}: Write down the kinetic
     and potential energy of the system and 
     write the dynamic model from there
    \item \textbf{Newton method}: Balance the forces and 
    write the dynamic model
\end{itemize}
\textbf{Newton method}:\\
The steps are the following:
\begin{enumerate}
    \item Write an equation for each body of the system
    \item Include in the equation the following forces:
    \item \begin{itemize}
        \item the force of the spring $k_i(x_i-x_{i-1})$
        \item the force of the damper $d_i(\dot{x}_i-\dot{x}_{i-1})$
        \item the force of the gravity $m_i g$
        \item the force of the acceleration $m_i \ddot{x}_i$
        \item the force of the external input $u_i$
        \item the force of the friction $f_i \dot{x}_i$
        \item the force of the contact $c_i$
    \end{itemize}
    \item Obtain something like: \begin{equation}
        m_i \ddot{x}_i + k_i(x_i-x_{i-1}) + d_i(\dot{x}_i-\dot{x}_{i-1}) + m_i g + f_i \dot{x}_i + c_i = u_i
    \end{equation}
\end{enumerate}

\subsubsection{ex: Find the torque $\tau$ that realizes given objectives}
The steps are the following:
\begin{enumerate}
    \item Write the dynamic model of the robot
    \item Isolate the torque $\tau$ from the dynamic model
    \item If asked to minimize the squared norm of 
    joint accelerations $H=1/2||\ddot{q}||^2$:
    \begin{itemize}
    \item find the value of $\ddot{q}$ to substitute 
    in the dynamic model as: \begin{equation}
    \ddot{q} = J(q)^{\#}(\ddot{r}_d + \dot{J}(q)\dot{q})+ (I-J(q)^{\#}J(q))\ddot{q}_0
    \end{equation}
    \end{itemize}
    \item If asked to minimize the squared norm of 
    absolute joint accelerations (joint acelerations but in absolute 
    coordinates) $H=1/2||\ddot{q_a}||^2$:
    \begin{itemize}
        \item Find the transformation matrix $T$ such that 
        $\ddot{q_a}=T\ddot{q}$
        \item Common example in a RRR robot is: \begin{equation}
        T = \begin{bmatrix}
            1 & 0 & 0 \\
            1 & 1 & 0 \\
            1 & 1 & 1
        \end{bmatrix}
        \end{equation}
        \item Compute the weighting matrix $W$ for the pseudoinverse 
        as $W=T^TT$
        \item Find the value of $\ddot{q_a}$ to substitute 
        in the dynamic model using the weighted pseudoinverse:
        \begin{equation}
            \ddot{q} = J(q)_W^{\#}(\ddot{r}_d + \dot{J}(q)_w\dot{q})+ (I-J(q)^{\#}_wJ(q)_w)\ddot{q}_0
            \end{equation}
            
    \end{itemize}
    \item If asked to minimize the squared norm of 
    the inertia weighted 
    joint accelerations $H=1/2\ddot{q}^TM(q)\ddot{q}$:
    \begin{itemize}
        \item Use the inertia weighted pseudoinverse
        \item Find the value of $\ddot{q}$ to substitute 
        in the dynamic model using the inertia weighted pseudoinverse:
        \begin{equation}
            \ddot{q} = J(q)_M^{\#}(\ddot{r}_d + \dot{J}(q)_M\dot{q})+ (I-J(q)^{\#}_MJ(q)_M)\ddot{q}_0
            \end{equation}
    \end{itemize}
\end{enumerate}





\section{APPENDIX}
\subsection{Pseudo-inverse properties}\label{sec:Pseudo-inverse properties}
The properties of the pseudo-inverse are:
\begin{itemize}
    \item $J J^\# J = J$
    \item $J^\# J J^\# = J^\#$
    \item $(J J^\#)^T = J J^\#$
    \item $(J^\# J)^T = J^\# J$
\end{itemize}
\subsection{Weighted pseudo-inverse properties}
The properties of the weighted pseudo-inverse are:
\begin{itemize}
    \item $J J^\# J = J$
    \item $J^\# J J^\# = J^\#$
    \item $(J J^\#)^T = J J^\#$
\end{itemize}
\subsection{DH frames}
\subsubsection{Assign axis}
\begin{itemize}
    \item $z_i$ along the direction of joint i+1
    \item $x_i$ along the common normal between $z_{i-1}$ and $z_i$
    \item $y_i$ completing the right-handed frame
\end{itemize}
\subsubsection{DH table}
\begin{itemize}
    \item $\alpha_i$ is the angle from $z_{i-1}$ to $z_i$ about $x_i$
    \item $a_i$ is the distance from $z_{i-1}$ to $z_i$ along $x_i$
    \item $d_i$ is the distance from $x_{i-1}$ to $x_i$ along $z_{i-1}$
    \item $\theta_i$ is the angle from $x_{i-1}$ to $x_i$ about $z_{i-1}$
\end{itemize}
\subsection{Trajectories} \label{sec:Trajectories}
\subsubsection{Cubic Polynomial}
The cubic polynomial is defined as:
\begin{equation}
    q(\tau)= a\tau^3+b\tau^2+c\tau+d
\end{equation}
We further define the \textbf{doubly normalized polyinomial} as:
\begin{equation}
    q_N(\tau)= q_0 + \Delta(q)(a\tau^3+b\tau^2+c\tau+d)
\end{equation}
\textbf{Special case} if both initial and final velocity are zero (rest-to-rest) we have:
\begin{equation}
    q(\tau)= q_0 + \Delta(q)(-2\tau^3+3\tau^2)
\end{equation}
\subsubsection{Quintic Polynomial}
The quintic polynomial is defined as:
\begin{equation}
    q(\tau)= a\tau^5+b\tau^4+c\tau^3+d\tau^2+e\tau+f
\end{equation}
We further define the \textbf{doubly normalized polyinomial} as:
\begin{equation}
    q_N(\tau)= q_0 + \Delta(q)(a\tau^5+b\tau^4+c\tau^3+d\tau^2+e\tau+f)
\end{equation}
\textbf{Special case} if $v_{in}=v_{fin}=0$ and $a_{in}=a_{fin}=0$ (rest-to-rest) we have:
\begin{equation}
    q(\tau)=q0 + \Delta(q)(6\tau^5-15\tau^4+10\tau^3)
\end{equation}
\subsection{Non-smooth trajectories}
\subsubsection{Bang-Bang Trajectory}
The bang-bang trajectory is defined as:
\begin{equation}
    q(t) = \begin{cases}
        \frac{a_{max}t^2}{2} & t \in [0,t_s)\\
        -\frac{a_{max}(T-t)^2}{2}+v_{max}T -\frac{v^2_{max}}{a_{max}} & t \in [T-t_s,T]
    \end{cases}
\end{equation}
\begin{equation}
    \dot{q}(t) = \begin{cases}
        a_{max}t & t \in [0,t_s)\\
        -a_{max}(T-t_s)+v_{max} & t \in [T-t_s,T]
    \end{cases}
\end{equation}
\begin{equation}
    \ddot{q}(t) = \begin{cases}
        a_{max} & t \in [0,t_s)\\
        -a_{max} & t \in [T-t_s,T]
    \end{cases}
\end{equation}
\subsubsection{ex: Finding minimum time}
To find the minimum time given max velocity and max acceleration we need to find the max
of all the possible time computation for the bang-bang trajectory.
\begin{enumerate}
    \item Compute the total distance or length (L or $\Delta(q)$) fro each joint
    \item Compute the possible values of time (for each joint) as:
            $T_{min} = \sqrt{\frac{4L}{a_{max}}}$
    \item Note: We can do that if we have initial and final velocity equal to zero
    \item (we obtain that from $v^2=v^2_0+2a\Delta(q)$)
    \item Find the maximum value of the possible values of time
\end{enumerate}
\subsubsection{Bang-Coast-Bang Trajectory}
The bang-coast-bang trajectory is defined as:
\begin{equation}
        q(t) = \begin{cases}
            \frac{a_{max}t^2}{2} & t \in [0,t_s)\\
            v_{max}t - \frac{v^2_{max}}{2a_{max}} & t \in [t_s,T-t_s)\\
            -\frac{a_{max}(T-t_s)^2}{2}+v_{max}T -\frac{v^2_{max}}{a_{max}} & t \in [T-t_s,T]
        \end{cases}
    \end{equation}
    \begin{equation}
        \dot{q}(t) = \begin{cases}
            a_{max}t & t \in [0,t_s)\\
            v_{max} & t \in [t_s,T-t_s)\\
            -a_{max}(T-t_s)+v_{max} & t \in [T-t_s,T]
        \end{cases}
    \end{equation}
    \begin{equation}
        \ddot{q}(t) = \begin{cases}
            a_{max} & t \in [0,t_s)\\
            0 & t \in [t_s,T-t_s)\\
            -a_{max} & t \in [T-t_s,T]
        \end{cases}
\end{equation}
\subsubsection{ex: Finding $t_s$ and $t_c$}
To find the values of $t_s$ and $t_c$ we need to impose the following constraints:
\begin{equation}
t_s = \frac{v_{max}}{a_{max}} 
\end{equation}
\begin{equation}
    t_c = \frac{La_{max} + v^2_{max}}{a_{max}v_{max}} - 2*t_s
    \end{equation}
 The final time is given by $t_f = t_c + 2t_s$

\subsubsection{ex: Finding minimum time}
To find the minimum time given max velocity and max acceleration we need to find the max 
of all the possible time computation for the bang-coast-bang trajectory.
\begin{enumerate}
    \item Compute the total distance or length (L or $\Delta(q)$) fro each joint
    \item Compute the possible values of time (for each joint) as: \begin{itemize}
        \item $
            T_{min} = \frac{La_{max}+v^2_{max}}{a_{max}v_{max}}$
        \item $
            T_{min} = \frac{L}{v_{max}}$
    \end{itemize}
    \item Find the maximum value of the possible values of time
\end{enumerate}
\subsection{Energy}
\subsubsection{Kinetic energy}
The kinetic energy of a spring is given by:
\begin{equation}
    T = \frac{1}{2} (x_1-x_2)^T K (x_1-x_2)
\end{equation}
The kinetic energy of a body accelerated by an acceleration $\ddot{q}$
is given by:
\begin{equation}
    T = \frac{1}{2} \dot{q}^T M \dot{q}
\end{equation}
\textbf{Note:} when adding those terms in 
the Lyapunov function we need to consider the
eventual control virtual springs Kp and Kd.

\subsubsection{Potential energy}
The potential energy is given by gravity.

\subsection{Laplace stability}
\begin{itemize}
    \item The system is stable if all the poles of the 
    transfer function have negative real part i.e. the poles lie in the 
    left half plane
\end{itemize}
\subsubsection{Transfer function}
The transfer function is defined as:
\begin{equation}
    G(s) = \frac{Y(s)}{U(s)}
\end{equation}
where $Y(s)$ is the output and $U(s)$ is the input
\subsubsection{Poles}
The poles of the transfer function are the values of $s$ 
that make the denominator of the transfer function equal to zero
\
\subsection{Routh Hurwitz criterion}
The Routh Hurwitz criterion is a necessary and sufficient
 condition for the stability of a system.\\
    The criterion is based on the coefficients of the characteristic
        polynomial of the system.\\
        The criterion states that a system is stable if and only if all the 
        coefficients of the characteristic polynomial are positive.\\
        The criterion is based on the construction of a table called the 
        Routh table.\\
        The Routh table has $n+1$ (where $n$ is the maximum power) 
        rows and is constructed as follows:
        \begin{itemize}
            \item The first row of the table is the coefficients of the 
            characteristic polynomial (of the even powers)
            \item The second row is the coefficients of the characteristic
            polynomial (of the odd powers)
            \item The other rows are computed as follows:
            \begin{itemize}
                \item The first element of the row is computed as:
                \begin{equation}
                    a_{i,1} = -\frac{det\begin{bmatrix}
                        a_{i-2,1} & a_{i-2,2}\\
                        a_{i-1,1} & a_{i-1,2}
                    \end{bmatrix}}{a_{i-1,1}}
                \end{equation}
                \item The other elements of the row are computed as:
                \begin{equation}
                    a_{i,j} = -\frac{det\begin{bmatrix}
                        a_{i-2,j-1} & a_{i-2,j}\\
                        a_{i-1,j-1} & a_{i-1,j}
                    \end{bmatrix}}{a_{i-1,1}}
                \end{equation}
            \end{itemize}
        \end{itemize}
        The system is stable if and only if all the elements
         of the first column
        of the Routh table are positive.





\section{GLOSSARY}
\begin{itemize}
\item \textbf{Self-motion}: Motion of a robot that does 
not change the position of the end-effector
\label{glossary:self-motion}
\item \textbf{Nominal condition}: The nominal condition is the condition 
in which the robot is supposed to be in.
\label{glossary:nominal condition}
\end{itemize}

\end{document}